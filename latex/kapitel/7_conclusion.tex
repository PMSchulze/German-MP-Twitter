\section{Conclusion}
\label{Conclusion}

Today, information from a wide variety of fields are publicly available on social media and various other forms of online appearances. Using techniques such as web scraping, the data, which in many cases consists of text, is readily obtained. In order to extract the information contained within these data, a proper analysis of large-scale unstructured text is, in many cases, a central task. Within this realm, topic modeling plays an important role.

In this paper we applied the Structural Topic Model (STM) to a large data set of Twitter messages and meta-information associated with members of the German Bundestag. In a first step, examining the estimated proportions of topics provided a concise summary of the predominant themes contained within the Tweets. Moreover, relating topic proportions to metadata in a descriptive manner allowed us to explore the topical structure with respect to different dimensions, such as time or the membership in a political party. Beyond such explorative analyses, in order to determine cause-effect relationships between metadata covariates and topics, we presented a train-test split framework for the STM, which was recently developed by \cite{egami2018make}. Extending the traditional topic modeling framework in order to examine causality between estimated topics and metadata is a challenging task and a current field of research. 

Throughout our analyses we have put special emphasis on the statistical assumptions and properties of the STM. While our comparison between the STM and the CTM confirms that metadata clearly has an influence on the estimated topics, this influence seems to be rather small in general. Nevertheless, we believe that the STM leverages document-specific characteristics, resulting in an estimated topical structure which is likely to be more realistic than is the case when employing models that do not consider metadata information. This is also reflected by a higher heldout-likelihood of the STM when compared to simpler topic models, as shown by \cite{roberts2016model}. 

When the explicit aim is to investigate the relationship between metadata and topics, aside from these potential improvements in accuracy, the advantages of the STM are less obvious. As with other topic models, the estimation of such relationships occurs in a separate, second step. That is, the STM (and especially its implementation in the R package \textit{stm}) does not directly produce a usable estimate of the relation between metadata and topics. Instead, in order to estimate such relations the authors of the STM employ a sampling procedure also known as the method of composition, which is implemented through the function \textit{estimateEffect} in the \textit{stm} package. Within this approach, sampled topic proportions are regressed on metadata covariates using an ordinary least squares (OLS) regression. We have demonstrated several shortcomings of this approach and presented possible alternatives. First, when dealing with (sampled) topic proportions, it has to be considered that these proportions are restricted to the interval $(0,1)$. By using regression approaches that assume a dependent variable in $(0,1)$, we extended the method of composition within the framework of the STM. Furthermore, separately modeling topic proportions, as is the case with \textit{estimateEffect}, is a vast simplification, since interdependence among different topics is neglected. We demonstrated this shortcoming by directly assessing the estimated covariance structure of prevalence covariates in section \ref{Direct assessment}. In our view, the results obtained by this direct assessment are more realistic than the results achieved using the method of composition. 

When examining causal effects beyond the exploration of topic-metadata relationships, we suggest to perform a train-test split. Conducting both steps on the same data, i.e., the estimation of topics and the subsequent estimation of effects based on these topics, results in a biased estimation of these effects. As discussed in section \ref{Causal Inference: Train-test Split}, the STM is well-suited for a train-test framework, since it allows including information contained within the metadata of the training set when predicting topic proportions on the test set. This is a clear benefit of the STM, emerging from the more advanced design of the STM compared to other topic models such as LDA.