\documentclass[12pt]{article}
%\usepackage[latin1]{inputenc}
\usepackage[T1]{fontenc}
\usepackage{geometry}
\usepackage{graphicx}
\usepackage{subcaption}
\usepackage[onehalfspacing]{setspace}
\usepackage{amsmath,amsfonts,amssymb,amsthm}
\usepackage{bm}
\usepackage{commath}
\usepackage{enumerate}
\usepackage{accents}
%\usepackage{enumitem}
\usepackage[shortlabels]{enumitem}
\usepackage{dsfont}
\usepackage{mathtools}
\usepackage{physics}
\usepackage{cite}
\usepackage[round]{natbib}
\usepackage{caption}
\captionsetup[figure]{font=small}
\usepackage{float}
%\usepackage{hyperref}
\usepackage{bbm}
\usepackage{longtable}

\newtheorem*{theorem}{Theorem} 
\newtheorem*{lemma}{Lemma}
\newtheorem*{definition}{Definition}
\newtheorem*{corollary}{Corollary}
\newtheorem*{remark}{Remark}
\newtheorem*{example}{Example}
\newtheorem*{examples}{Examples}
\newcommand*{\QEDB}{\hfill\ensuremath{\square}}

\DeclareMathOperator*{\argmax}{arg\,max}
\DeclareMathOperator*{\argmin}{arg\,min}

\newcommand\independent{\protect\mathpalette{\protect\independenT}{\perp}}
\def\independenT#1#2{\mathrel{\rlap{$#1#2$}\mkern2mu{#1#2}}}

\interfootnotelinepenalty=10000
\allowdisplaybreaks
\geometry{
  left=2.5cm,
  right=2.5cm,
  top=2cm,
  bottom=2cm,
}

\begin{document}
\numberwithin{equation}{section}
\begin{titlepage}
	\centering
	{\scshape\LARGE Ludwig-Maximilians-Universit\"at M\"unchen \par}
	\vspace{1cm}
	{\scshape\Large Statistical Consulting Project \par}
	\vspace{1.5cm}
	{\huge\bfseries Twitter in the Parliament - A Text-based Analysis of German Political Entities  \par}
	\vspace{2cm}
	{\Large\itshape Patrick Schulze, Simon Wiegrebe \par}
	\vspace{2cm}	
	\abstract{
\noindent
The analysis of large-scale unstructured data is gaining importance at both a professional and academic level, certainly helped by the omnipresence of social media and novel statistical analysis tools. For instance, topic models seek to discover latent thematic clusters within text data. Social scientists increasingly move beyond a merely explorative use of such topic models, focusing on relationship estimation and causal inference. Oftentimes, however, the statistical subtleties resulting from the latency of topics are not well understood or insufficiently addressed. We construct a dataset containing German parliamentarians’ Twitter posts as well as a variety of document metadata and explore it by means of the Structural Topic Model (STM). Subsequently, we address the relationship between topic proportions and metadata, providing enhanced analytical tools with improved statistical properties and applying a train-test split to facilitate causal inference.
	}
	\vfill
	\begin{center}
	supervised by\par
	Prof. Dr. Christian Heumann and Prof. Dr. Paul W. Thurner

	\vfill

	{\large \today\par}
	\end{center}
\end{titlepage}

\tableofcontents
\newpage
\section{Introduction}

The rise in popularity of social media has changed various aspects of private, public, and professional life over the last two decades. From a data-analytical point of view, this has led to an unprecedented increase in the supply of publicly available unstructured (text) data, ready to be analyzed. In fact, unstructured data makes up the lion's share of what is called \textit{big data} (\citealp{gandomi2015beyond}). At the same time, advances in the field of machine learning, particularly in \textit{Natural Language Processing} (NLP), have created numerous new opportunities for the analysis of such large-scale unstructured texts.

A field which has been particularly impacted by the use of social media (and the information extracted from it) is politics. At least since the 2016 Brexit vote and US presidential election, politicians have come to recognize not only that social media presence is ever more important, but also how strong a message their social media behavior can transmit. Among social media networks, Twitter is of particular importance, since it allows for direct communication between politicians and voters - and even more so after the Facebook-Cambridge Analytica data breach in 2018. As a consequence, there has been increasing academic interest in text-based (intra- and inter-)party politics (e.g., \citealp{ceron2017intra, daniel2019static, grimmer2010bayesian, quinlan2018show}). Moreover, unstructured text and the insights generated from it can subsequently be used as input for a broad variety of tasks, ranging from election forecasts (e.g., \citealp{burnap2016140, jungherr2016twitter, tumasjan2010predicting}) to prediction of stock market movements (e.g., \citealp{nisar2018twitter}).

A key challenge in analyzing large amounts of unstructured text is to reduce dimensionality and classify pieces of text: either into previously determined categories (for instance, sentiments), which corresponds to a supervised learning problem; or by trying to discover latent thematic clusters that govern the content of the documents, which is now an instance of unsupervised learning (since the number and labeling of clusters is to be determined). In this paper, we use a mixture of both strategies - an unsupervised \textit{topic model} followed by supervised regression analysis - and apply it to German politics. In particular, we construct a dataset where the text documents consist of Twitter messages by German Members of Parliament (MPs) and which furthermore contains a plenitude of personal MP-level data as well as socioeconomic data on an electoral-district level. Subsequently, we fit a \textit{Structural Topic Model} (STM) to the data to discover latent topics and analyze their relationship with document-level metadata. The focus of this paper lies on statistical and methodological aspects of topic models in general, but especially regarding the relation of topics to metadata, instead of specific (politological) hypothesis testing. Altogether, thus, the contribution of this paper is threefold: first, a broad and widely applicable dataset for future research, particularly in political science; second, a topic analysis of German parliamentarians' Twitter communication; and third, critical discussion of existing and development of new tools for (causal) inference within a topic modeling context.

We find that for most model specifications the majority of topics carry meaning, which can be regarded as a form of retrospective model validation. The fact that these topics are converted from mere word clusters into actually meaningful thematic clusters through manual labeling underlines the importance of human judgment in statistical topic modeling - this is in line with \cite{chang2009reading}, who show that solely focusing on quantitative metrics such as held-out likelihood does not guarantee meaningfulness of the latent space. As for document-level metadata, we discover some relevant associations between topic proportions and document features; particularly for the political parties, these relationships are in line with expectation. For continuous covariates such as unemployment and GDP the high degree of uncertainty induced by the underlying generative process of the STM renders relationships insignificant - though the observed tendencies are consistent across all modeling methodologies. The inclusion of a covariate to further model topical content (beyond its effect on topical prevalence) is found to reduce the meaningfulness of the latent space; furthermore, no natural candidate for the topical content variable exists in our case. Finally, we find that double usage of (prevalence) covariate information does not pose a problem, while double usage of document-level associations induces a substantial degree of overfitting.

The remainder of this paper is organized as follows. Section 2 provides the theoretical foundation of topic modeling, in particular the "component models" of the STM which we use for the major part of our analysis, as well as a brief discussion of inference and parameter estimation. Section 3 describes the data collection process, the data itself, and the data preprocessing necessary for topic modeling. Section 4 discusses model selection, labeling as well as global characteristics of the latent space. In section 5, we include document-level metadata into the analysis, presenting the corresponding theory and results. Section 6 deals with alternative modeling approaches and strategies for causal inference. Finally, section 7 concludes.
\section{Theoretical Framework}
\label{Theoretical Framework}

\subsection{Topic Modeling - Overview}
\label{Topic Modeling - Overview}

Topic models seek to discover latent thematic clusters, called topics, within a collection of discrete data, usually text; therefore, topic modeling can be regarded as dimensionality reduction technique. Furthermore, since both the number and content of topics is unknown beforehand (and can never be truly verified), topic modeling is an instance of unsupervised learning. Information retrieval (IR) research generally proposes the reduction of text documents to vectors of real numbers, each number representing (modified) counts of or terms. An instance of this proposed methodology is the \textit{tf-idf} scheme by \cite{salton1983information}, which for a collection of documents returns a term-by-document matrix where each row corresponds to a document in the corpus and the columns contain the respective \textit{tf-idf} term count. Since only words in a vocabulary of fixed length $V$ are considered, documents of unrestricted length are being reduced to vectors of a fixed length $V$. To further reduce dimensionality, the \textit{latent semantic indexing} (LSI) by \cite{deerwester1990indexing} applied singular value decomposition (SVD) to the \textit{tf-idf} document-term matrix. However, as \cite{blei2003latent} argue, the idea should be to develop a generative probabilistic model of text, in order to estimate to which extent the LSI methodology can align data with the generative text model; yet, given such a model, Bayesian methods or maximum likelihood estimation (MLE) would be much more direct, which is why the benefits of applying the LSI are not obvious. Picking up this shortcoming of LSI, \cite{hofmann1999probabilistic} introduced the \textit{probabilistic LSI} (pLSI) model. This generative data model allows for individual words to be sampled from a mixture model: they are drawn from a multinomial distribution, with latent random variables determining the mixture proportions, which in turn can be viewed as topics. However, the pLSI can only be regarded as partly probabilistic text model, since the mixing components themselves are fixed on a document level, thus lacking a probabilistic generating process.

In their \textit{Latent Dirichlet Allocation} (LDA) model, \cite{blei2003latent} included the generation of topic proportions into the generative probabilistic model, the resulting three-level hierarchical Bayesian mixture model marking the starting point of modern topic modeling. In order to present the main idea of LDA, we first introduce some notation and terminology that we will use throughout the remainder of this paper.

\begin{itemize}
\vspace{-0.25cm}
\item[•] A \textit{word} is the smallest unit of discrete text data. Words are instances of a vocabulary of $V$ unique \textit{terms} and can thus be indexed by $\{1,\dots,V\}$. Mathematically, the $v$-th term in the vocabulary can be represented as a vector of length $V$, whose $v$-th component equals one, with all other components equalling zero. We will sometimes refer to the $v$-th term of the vocabulary simply as $v$. Words, along with document-level metadata, represent the actually observable data.
\vspace{-0.25cm}
\item[•] A \textit{document} $d \in \{1,\dots,D\}$ is a sequence of words of length $N_{d}$. For a given document $d$, we denote its words by $(w_{d,1},\dots,w_{d,N_{d}})$. Consequently, the $n$-th word of document $d$ is denoted by $w_{d,n}$.
\vspace{-0.25cm}
\item[•] A \textit{corpus} is a collection (or set) of $D$ documents. Therefore, $d \in \{1,\dots,D\}$ means that our corpus contains $D$ documents.
\vspace{-0.25cm}
\item[•] A \textit{topic} is a latent thematic cluster within a text corpus. The idea is that any collection of documents is made up of $K$ such topics, where the number of topics $K$ is an (unknown) hyperparameter which needs to be determined ex ante (see section \ref{Hyperparameter Search and Model Fitting} for hyperparameter determination in our specific use case). We will refer to topics simply by the actual \textit{topic index} (or \textit{topic number}) $k \in \{1,\dots,K\}$.
\vspace{-0.25cm}
\item[•] A \textit{topic-word distribution} $\boldsymbol{\beta}$ is a probability distribution over words, i.e., over the vocabulary. For a model containing $K$ topics (and no topical content variable, see section \ref{The Structural Topic Model} below), topic-word distributions do not vary across documents and uniquely characterize a topic: we denote the word distribution corresponding to the $k$-th topic by $\boldsymbol{\beta}_k$ and the matrix whose $k$-th column is topic $\boldsymbol{\beta}_k$ by $\boldsymbol{B}:=\boldsymbol{\beta}_{1:K}=[\boldsymbol{\beta}_1|\dots|\boldsymbol{\beta}_K]$. Each vector $\boldsymbol{\beta}_k$ thus has length $V$, while $B$ is a $V \times K$-matrix. Therefore, $k$ refers to the latent thematic cluster with topic index $k$ in general, and $\boldsymbol{\beta}_k$ refers to the underlying word distribution in particular.
\vspace{-0.25cm}
\item[•] A \textit{topic assignment} $\boldsymbol{z}_{d,n}$ is the assignment of the $n$-th word of document $d$ to a specific topic $k \in \{1,\dots,K\}$ (i.e., to the corresponding word distribution $\boldsymbol{\beta}_k$). Therefore, $\boldsymbol{z}_{d,n}$ is simply a vector of length $K$ whose $k$-th entry equals one and all other entries equal zero. This allows us to represent the word distribution corresponding to the $n$-th word in document $d$ as $\boldsymbol{\beta}_{d,n}:=B\boldsymbol{z}_{d,n}$ (again, for a model without topical content variable).
\vspace{-0.25cm}
\item[•] For a given document $d$, the corresponding \textit{topic proportions}, denoted by $\boldsymbol{\theta}_d$, are the proportions of the document's terms assigned to each of the topics $k \in \{1,\dots,K\}$. Topic proportions vary across documents. Since for each document $d$ the proportions of all $K$ topics must add up to one ($\sum_{k=1}^{K}\theta_{d,k}=1$, for all $d \in \{1,\dots,D\}$), topic proportions represent probabilities.
\vspace{-0.25cm}
\item[•] The \textit{bag-of-word} assumption is an assumption used in all (probabilistic) text models referenced in this paper, including LSI and pLSI, and states that only words themselves (and their counts) carry meaning, while word order or grammar do not. Statistically, this is equivalent to assuming that words within a document are \textit{exchangeable} (\citealp{aldous1985exchangeability}).

\end{itemize}

As mentioned above, LDA is the first generative probabilistic model of an entire text corpus. (Recall that pLSI is only probabilisitic for a fixed document.) 
Now, the generative process underlying LDA can be described as a two-step procedure, where for each document $d \in \{1,\dots,D\}$:

\begin{enumerate}[{1)}]
\vspace{-0.25cm}
\item Draw topic proportions $\boldsymbol{\theta}_d \sim \text{Dir}_K(\boldsymbol{\alpha})$.
\vspace{-0.25cm}
\item For each word $n \in \{1,\dots,N_d\}$:
	\begin{enumerate}[{a)}]
	\vspace{-0.25cm}    
    \item Draw a topic assignment $\boldsymbol{z_{d,n}} \sim \text{Multinomial}_K(\boldsymbol{\theta}_d)$.
	\vspace{-0.25cm}    
    \item Draw a word $w_{d,n} \sim \text{Multinomial}_V(\boldsymbol{\beta}_{d,n})$.
	\end{enumerate}
\end{enumerate}

\noindent
Thus, topic proportions are drawn from a Dirichlet distribution with $K$-dimensional $\boldsymbol{\alpha}$, with all components $\alpha_k > 0$, where the number of topics $K$ is a hyperparameter to be set by the user; the $\boldsymbol{\alpha}$-vector is estimated from the data. This means that for each document $d \in \{1,\dots,D\}$, the corresponding topic proportions $\boldsymbol{\theta}_d$ represent a $K$-dimensional vector which can take on values on the ($K-1$)-simplex, i.e., $\theta_{d,k} \geq 0, \sum_{k=1}^{K}\theta_{d,k}=1$. Also note that the Dirichlet distribution is the conjugate prior of the multinomial distribution, which greatly facilitates estimation (see section \ref{Inference and Parameter Estimation} on variational inference below). Put simply, for each document LDA first generates topic proportions, which are then used as weights for topic assignment. A word is then drawn from a topic-specific word distribution, which is determined by the topic assignment. These topic-specific word distributions $\boldsymbol{\beta}_k$ need to be estimated from data.
Note that LDA is a very simple, restrictive model in (at least) three ways:

\begin{enumerate}[label=(\roman*)]
\vspace{-0.25cm}
\item By using the Dirichlet distribution to generate topic proportions, potential correlations between topics cannot be captured due to the neutrality of the Dirichlet distribution.\footnote{Due to the constraint $\sum_{k=1}^{K}\theta_{k}=1$, there is clearly some degree of dependence between topic proportions. However, the dependence is minimal, as the Dirichlet distribution is characterized by complete neutrality: the components $\theta_1/(1-S_0), \theta_2/(1-S_1),\dots, \theta_K(1-S_{K-1})$ are mutually independent, where $S_0:=0$ and $S_k = \sum_{i=1}^{k}\theta_k, k \in \{1,\dots,K\}$. Stated differently, for each component $\theta_k, k \in \{1,\dots,K\}$, it holds that $\theta_k/(1-S_{k-1})$ is independent of the vector constructed by weighting all \textit{remaining} components by their total proportion (\citealp{james1980new})}. As a consequence, the occurrence of one topic within a document is not correlated with the occurrence of another topic (\citealp{blei2007correlated}). This is a restrictive simplification, as topics such as "sports" and "health" are much more likely to co-occur within a document than, say, "sports" and "war".
\vspace{-0.25cm}
\item Second, while topic proportions vary stochastically across documents, they do so given a single, global hyperparameter vector $\boldsymbol{\alpha}$; recalling the bag-of-words assumption, this implies that topic proportions are generated based merely on word counts (occurrences and co-occurrences), while additional document-level information is not taken into account. This is another unrealistic and limiting simplification, since researchers usually possess further document-specific information indicative of the topics addressed within the individual documents.
\vspace{-0.25cm}
\item Third, by construction, the topic-specific word distributions $\boldsymbol{\beta}_k$ are assumed to be identical for all documents. Similarly to the second restriction, this prevents researchers from using (document-level) information which might potentially influence the weighting of specific words within a topic.

\end{enumerate}

\noindent
Due to its simplicity and the resulting restrictions, the LDA has been used used as a building block for more advanced (and usually more specified) generative topic models. One model that builds on LDA, addressing some of its shortcomings, is the \textit{Correlated Topic Model} (CTM) by \cite{blei2007correlated}. Specifically, the CTM addresses the first one of the abovementioned restrictions: the inability to cope with inter-topic correlations. The model no longer uses a Dirichlet distribution to sample topic proportions; instead, a logistic normal distribution is employed, which can capture correlations between topics due to the incorporated covariance structure between its components (\citealp{atchison1980logistic}). The resulting generative process for the CTM can be stated as follows:

\vspace{0.25cm}
\noindent
For each document $d \in \{1,\dots,D\}$:

\begin{enumerate}[{1)}]
\vspace{-0.25cm}
\item Draw unnormalized topic proportions $\boldsymbol{\eta}_d \sim \mathcal{N}_{K-1}(\boldsymbol{\mu}, \boldsymbol{\Sigma})$, with $\eta_{d,K} := 0$ for model identifiability.
\vspace{-0.25cm}
\item Normalize $\boldsymbol{\eta}_d$ by mapping it to the simplex: $\theta_{d,k} = \frac{exp(\eta_{d,k})}{\sum_{j=1}^{K}exp(\eta_{d,j})}$, for all $k \in \{1,\dots,K\}$.
\vspace{-0.25cm}
\item For each word $n \in \{1,\dots,N_d\}$:
	\begin{enumerate}[{a)}]
	\vspace{-0.25cm}    
    \item Draw a topic assignment $\boldsymbol{z_{d,n}} \sim \text{Multinomial}_K(\boldsymbol{\theta}_d)$.
	\vspace{-0.25cm}    
    \item Draw a word $w_{d,n} \sim \text{Multinomial}_V(\boldsymbol{\beta}_{d,n})$.
	\end{enumerate}
\end{enumerate}

\noindent
The first two steps constitute the sampling from a logistic normal distribution: a $K$-dimensional vector $\boldsymbol{\eta}_d$ is drawn from a multivariate normal distribution and subsequently transformed to a vector of proportions (or probabilities) by applying the \textit{softmax} function to each of its elements. The number of topics $K$ is again a hyperparameters which must be determined ex ante. As in LDA, the parameters of the normal distribution in step 1, $\boldsymbol{\mu} \in \mathbb{R}^{K-1}$ and $\boldsymbol{\Sigma} \in \mathbb{R}^{(K-1) \times (K-1)}$, as well as the topic-specific word distributions $\boldsymbol{\beta}_k$ need to be estimated from the data. As mentioned above, this process now allows for inter-topic correlation. Yet this comes at a cost: unlike the Dirichlet distribution, the logistic normal distribution is no longer conjugate to the multinomial distribution. As explained in more detail in section \ref{Inference and Parameter Estimation} below, this renders standard variational inference algorithms inapplicable, since these rely on conjugacy and the implied closed-form solutions. However, using the Laplace variational inference developed by \cite{wang2013variational}, which is a generic method for variational inference when dealing with nonconjugate models, solves the inference problem for the CTM.

As for the inability to integrate covariate information into the determination of topic proportions, \cite{mimno2011optimizing} were the first to model topic proportions as a function of \textit{observable} document-level metadata. Specifically, their \textit{Dirichlet-Multinomial Regression} (DMR) model still samples topic proportions $\boldsymbol{\theta}_d$ from a Dirichlet distribution (thus, not allowing for inter-topic correlations), yet unlike in LDA, the Dirichlet prior $\boldsymbol{\alpha}_d$ is no longer global but topic-specific. This topic prior $\boldsymbol{\alpha}_d$, in turn, is log-linear in the document-level features $\boldsymbol{x}_d$ and the (topic-specific) priors for the coefficients of these features, $\boldsymbol{\lambda}_t$, have a normal prior. With coefficients being updated through numerical optimization as part of the EM algorithm used for training, the DMR model thus actively uses document features to model topic proportions. 

Finally, the third restictiveness of LDA, the inflexibility of the topic-word distributions $\boldsymbol{\beta}_k$ when document-level metadata is available, is addressed by \cite{eisenstein2011sparse} in their \textit{Sparse Additive General} model (SAGE). The authors propose to start off with a background word distribution $m$ containing log frequencies and to model additive deviations from this baseline for each class. The idea behind SAGE can be used to model differences in topic-word distributions according to the category of some document-level covariate.

Based on the foundational LDA as well as its extensions, \cite{roberts2013structural} developed the \textit{Structural Topic Model} (STM), which combines the improvements over the original LDA discussed in this section. Due to its flexibility regarding the incorporation of document-level information, we choose the STM for our specific use case, a text-based analysis of German political entities (TBD, depends on final title of paper). Therefore, we discuss the model in greater detail in section \ref{The Structural Topic Model} below.

\subsection{The Structural Topic Model}
\label{The Structural Topic Model}

\subsubsection*{Overview}

The STM addresses the three main shortcomings of the LDA, as discussed in the previous section. In this subsection, we explain the corresponding modifications with respect to LDA and present the generative process of the STM.

\begin{enumerate}[label=(\roman*)]
\vspace{-0.25cm}
\item To allow for correlation among topics, the STM uses a logistic normal distribution to sample topic proportions. In fact, if no document-level metadata is fed into the STM, it simply reduces to the CTM.
\vspace{-0.25cm}
\item The STM allows for the incorporation and use of document-level metadata when determining topic proportions. Similar to the DMR, topic proportions $(\boldsymbol{\theta}_1,\dots,\boldsymbol{\theta}_D)^T$ are assumed to depend on $P$ document-level \textit{topical prevalence variables} (such as the author's name, her political party or her popularity on Twitter), yet now with each $\boldsymbol{\theta}_d$ following a multivariate logistic normal distribution. The distribution now has median vector $\boldsymbol{\Gamma}^T\boldsymbol{x_d}^T$\footnote{There is no analytical closed form for the mean in the logistic normal distribution.}; $\boldsymbol{X}=[\boldsymbol{x_1}|\dots|\boldsymbol{x_D}]^T \in \mathbb{R}^{D \times P}$ is a matrix containing $D$ document-level prevalence covariate vectors $\boldsymbol{x}_d$, each one containing $P$ document-level covariate values, $\boldsymbol{\Gamma} = [\boldsymbol{\gamma}_1|\dots|\boldsymbol{\gamma}_K]$ is a matrix with each of its $K$ columns $\boldsymbol{\gamma}_k$ being a $P$-dimensional vector of topic proportion coefficients for the respective topic $k$, and a matrix of parameters $\boldsymbol{\Sigma}$ determining the covariance structure (for details see further below). This way, the model accounts for the fact that document-level covariates might influence how much (that is, which percentage of the total number of words) the corresponding documents attribute to the different topics.
\vspace{-0.25cm}
\item Within the STM, document-level covariate information can also be used to fine-tune the topic-word distributions $\boldsymbol{\beta}_k$, the methodology being similar to the one in the SAGE model. In particular, the STM allows for specifying a single categorical document-level \textit{topical content variable} $\boldsymbol{Y}\in \mathbb{R}^D$ with $A$ levels, i.e., $Y_d \in \{1,\dots,A\}$, for all $d \in \{1,\dots,D\}$ (\citealp{stm}).\footnote{In theory, multiple topical content variables could be included, yet the R package \textbf{stm} (\citealp{stm}) only allows for specifying a single content variable due to computational complexity. Furthermore, $Y_d$ is dummy encoded in the model implementation, making $Y_d$ a vector of length $A$ and $\boldsymbol{Y}$ a matrix of dimension $D \times A$. However, for notational convenience we simply refer to $Y_d$ as a scalar and to $\boldsymbol{Y}$ as a $D$-dimensional vector here.} Consequently, each topic $k \in \{1,\dots,K\}$ is now associated with a total of $A$ topic-word distributions $\boldsymbol{\beta}_{k,a}, a \in \{1,\dots,A\}$ instead of a single one, $\boldsymbol{\beta}_k$. For a given document $d$, this means the $K$ topic-word distributions are now additionally determined by the level $a$ assumed by $Y_d$ and are identical across all documents with $Y_d = a$, given a topic $k \in \{1,\dots,K\}$ (\citealp{roberts2016model}). This way, for a given document $d$, document-level metadata can not only impact the weighting of topics $\boldsymbol{\theta}_d$, but also the weighting over words for each topic. Note that for a given topic $k$, the word distributions $\boldsymbol{\beta}_{k,a}$ do not vary substantially across different values of $a$; that is, the content variable $\boldsymbol{Y}$ is really an $A$-level refinement of $\boldsymbol{\beta}_k$ and does \textit{not} affect the number of topics $K$.

\end{enumerate}

\noindent
The generative process of the STM can be stated as follows (\citealp{roberts2016model}):

\vspace{0.25cm}
\noindent
For each document $d \in \{1,\dots,D\}$:

\begin{enumerate}[{1)}]
\vspace{-0.25cm}
\item Draw unnormalized topic proportions $\boldsymbol{\eta}_d \sim \mathcal{N}_{K-1}(\boldsymbol{\Gamma}^T\boldsymbol{x_d}^T, \boldsymbol{\Sigma})$, with $\eta_{d,K}$ set to zero for model identifiability.
\vspace{-0.25cm}
\item Normalize $\boldsymbol{\eta}_d$ by mapping it to the simplex: $\theta_{d,k} = \frac{exp(\eta_{d,k})}{\sum_{j=1}^{K}exp(\eta_{d,j})}$, for all $k \in \{1,\dots,K\}$.
\vspace{-0.25cm}
\item For each word $n \in \{1,\dots,N_d\}$:
	\begin{enumerate}[{a)}]
	\vspace{-0.25cm}    
    \item Draw a topic assignment $\boldsymbol{z_{d,n}} \sim \text{Multinomial}_K(\boldsymbol{\theta}_d)$.
	\vspace{-0.25cm}    
    \item If no topical content variable has been specified, simply draw a word $w_{d,n} \sim \text{Multinomial}_V(\boldsymbol{\beta}_{d,n})$. Otherwise, first determine the document-specific word distributions $B_a := [\boldsymbol{\beta}_{1,a}|\dots|\boldsymbol{\beta}_{K,a}]$ based on the level $a$ taken on by $Y_d$, for all topics $k \in \{1,\dots,K\}$; next, analogously define $\boldsymbol{\beta}_{d,n}:=B_a\boldsymbol{z_{d,n}}$; finally, draw a word $w_{d,n} \sim \text{Multinomial}_V(\boldsymbol{\beta}_{d,n})$.
	\end{enumerate}
\end{enumerate}

\noindent
This means that unnormalized topic proportions are sampled from a normal distribution with mean $\boldsymbol{\Gamma} = [\boldsymbol{\gamma}_1|\dots|\boldsymbol{\gamma}_K]$ and covariance $\boldsymbol{\Sigma}$. The coefficients contained in the matrix $\boldsymbol{\Gamma}$, which correspond to the prevalence covariates $\boldsymbol{X}$, are obtained through a Bayesian linear regression with prior distributions $\boldsymbol{\gamma}_k \sim \mathcal{N}_p(0, \sigma_k^2\boldsymbol{I}_p)$ (see \citealp{roberts2016model} for further details). However, we actually specify an L1 penalty instead of this prior to ensure model convergence. The softmax function is then applied to unnormalized topic proportions $\boldsymbol{\eta}_d$, yielding normalized topic proportions $\boldsymbol{\theta}_d$, which in turn are used as weights for the subsequent topic assignment $\boldsymbol{z_{d,n}}$. Finally, each word is sampled from the corresponding multinomial word probability distribution (over the vocabulary of length $V$), which depends on topic assignment $\boldsymbol{z_{d,n}}$ and, for models containing a topical content variable, on its level $a$. In line with SAGE methodology, the topic-word distributions are modelled as deviations in log-frequency from a baseline vocabulary. (See \citealp{roberts2016model}, p.\ 991 for more details.) As in the CTM, $K$ (and $\sigma_k^2$ if no L1 penalty is specified) is a hyperparameter to be chosen by the user. The graphical model representation in Figure 1 below visualizes the generative process described.


\begin{figure}[h!]
  \centering
  \captionsetup{justification=centering,margin=2cm}
  \includegraphics[scale = 0.5]{../plots/2/stm_graphical.png}
  \caption{Graphical model representation of the STM (from \cite{roberts2016model}, p.\ 990).}
  \label{fig:graphical_model}
\end{figure}

\subsubsection*{Scope}

Topic models are unsupervised learning methods, since the true topics from which the text was generated are not known. Thus, topic models have been traditionally used as an exploratory tool providing a concise summary of topics, with the posterior ideally inducing a good decomposition of the corpus. Topic models have also been used for tasks such as collaborative filtering and classification (see, e.g.,\ \citealp{blei2003latent}). In particular, they can be used as dimensionality reduction method in semi-supervised learning methods. Such a process can in general be described as a two-stage approach, where in the first stage topic proportions and content are learned, and in the second stage a supervised method such as regression takes this learned representation as input. 

The fundamental idea of the STM is to combine these two steps: Topics and their association with covariates are estimated jointly. For instance, the estimated effect of topical prevalence covariates $\boldsymbol{X}_d$ on topic proportions $\boldsymbol{\theta}_d$ is reflected in the estimate of $\boldsymbol{\Gamma}$. However, since the topic proportions are latent random variables, it is preferable to incorporate the uncertainty of $\boldsymbol{\theta}_d$, accesible through the estimated approximation of the posterior $p(\boldsymbol{\theta}_d | \boldsymbol{\Gamma}, \boldsymbol{\Sigma}, \boldsymbol{X})$, when determinig the effect of covariates on topic proportions. This is achieved by what is called the "method of composition" in social sciences: By sampling from the approximate posterior for $\boldsymbol{\theta}$ and subsequently regressing these topic proportions on $\boldsymbol{X}$, it is possible to integrate out the topic proportions (since these are latent variables) and obtain an i.i.d.\ sample from the marginal posterior of the regression coefficients for the topical prevalence covariates (see Section \ref{Metadata Analysis - Topical Prevalence and Content}).

A problem we see with this approach, however, is that the same covariates - and in general the same data - used to infer the topical structure are subsequently used to determine effects of the former on the latter (or vice versa). This problem has recently also been adressed by \cite{egami2018make}. In our specific case, we find that the prevalence covariates do not have much impact on the estimated topic proportions due to the regularizing priors for $\boldsymbol{\Gamma}$ (see Section \ref{Overfitting and Causal Inference - Alternative Modeling Strategies}). Thus, the regression coefficients (with topic proportions as the dependent variable) should not be largely affected by this problem of double usage. However, this begs the question why document-level covariates are being used to obtain the topical structure in the first place. In an empirical evaluation, \cite{roberts2016model} showed that the STM consistently outperforms other topic models, such as LDA, when comparing the respective heldout likelihoods in different settings. This indicates that the STM performs better at predicting the topical structure by incorporating covariates, regardless of their concrete specification.

Nevertheless, in each case it should be investigated whether the relationship of variables implied by the STM is valid. In line with \cite{egami2018make}, we address this issue in section \ref{Causal Inference: Train-test Split}, where we split our data into a training and a test set. Similar topical structures on both datasets (as we find in our case) indicate that misspecification of topical prevalence or content variables is not a concern. However, since the topical prevalence covariates have almost no influence on the estimated topic proportions on the training set due to the regularizing priors (and likewise on the heldout likelihood that can be used for validation), it is practically impossible to validate a good prevalence specification.

\subsubsection*{Posterior Distribution}

In this subsection, we briefly derive the posterior distribution of the STM (up to proportionality), as stated on p.\ 992 of \cite{roberts2016model}. Recall that only words $w$, prevalence covariates $\boldsymbol{X}$, and the content covariate $\boldsymbol{Y}$ are observable, while unnormalized topic proportions $\boldsymbol{\eta}$ and topic assignments $\boldsymbol{z}$ are latent and topic-word distribution deviations $\boldsymbol{\kappa}$, prevalence coefficients $\boldsymbol{\Gamma}$, and unnormalized topic proportion variance $\boldsymbol{\Sigma}$ - are parameters to be estimated.
\begin{align*}
p(\boldsymbol{\eta}, \boldsymbol{z}, \boldsymbol{\kappa}, \boldsymbol{\Gamma}, \boldsymbol{\Sigma} | \boldsymbol{w}, \boldsymbol{X}, \boldsymbol{Y}) & \propto \underbrace{p(\boldsymbol{w} | \boldsymbol{\eta}, \boldsymbol{z}, \boldsymbol{\kappa}, \boldsymbol{\Gamma}, \boldsymbol{\Sigma}, \boldsymbol{X}, \boldsymbol{Y})}_{=p(\boldsymbol{w} | \boldsymbol{z}, \boldsymbol{\kappa}, \boldsymbol{Y})} p(\boldsymbol{\eta}, \boldsymbol{z}, \boldsymbol{\kappa}, \boldsymbol{\Gamma}, \boldsymbol{\Sigma} | \boldsymbol{X}, \boldsymbol{Y}) \\
& \propto p(\boldsymbol{w} | \boldsymbol{z}, \boldsymbol{\kappa}, \boldsymbol{Y}) p(\boldsymbol{z} | \boldsymbol{\eta}) p(\boldsymbol{\eta} | \boldsymbol{\Gamma}, \boldsymbol{\Sigma}, \boldsymbol{X}) \prod p(\boldsymbol{\kappa}) \prod p(\boldsymbol{\Gamma}) \\
& \propto \Big\{ \prod_{d=1}^{D} p(\boldsymbol{\eta}_d | \boldsymbol{\Gamma}, \boldsymbol{\Sigma}, \boldsymbol{X}_d) \Big( \prod_{n=1}^{N} p(w_{d,n} | \boldsymbol{\beta}_{d, n}) p(\boldsymbol{z_{d,n}} | \boldsymbol{\theta}_d) \Big) \Big\} \prod p(\boldsymbol{\kappa}) \prod p(\boldsymbol{\Gamma}) \\
& \propto \Big\{ \prod_{d=1}^{D} \text{Normal}(\boldsymbol{\eta}_d | \boldsymbol{X}_d \boldsymbol{\Gamma}, \boldsymbol{\Sigma}) \Big( \prod_{n=1}^{N} \text{Multinomial}(\boldsymbol{z}_{n,d}| \boldsymbol{\theta}_d) \\
& \ \ \ \ \times \text{Multinomial}(w_{d,n} | \boldsymbol{\beta}_{d,n}) \Big) \Big\} \times \prod p(\boldsymbol{\kappa}) \prod p(\boldsymbol{\Gamma}),
\end{align*}
where $\boldsymbol{\beta}_{d, n} \in \mathbb{R}^V$ is the topic-word distribution for word $n$ in document $d$, which has been assigned to topic $k$ through $\boldsymbol{z_{d,n}}$. The topic-word distribution vectors $\boldsymbol{\beta}_{k,a}$ have entries $\beta_{k,a,v} \propto \exp(m_{v} + \kappa_{k,v}^{(t)} + \kappa_{a,v}^{(c)} + \kappa_{k, a,v}^{(i)})$, $v \in \{1,\dots,V\}$, where $\kappa_{k,v}^{(t)}$, $\kappa_{a,v}^{(c)}$, and $\kappa_{k, a,v}^{(i)}$ are the log-transformed rate deviations of word $v$ for topic $k$, for content variable level $a$, and for the interaction of $k$ and $a$, respectively.

\subsection{Inference and Parameter Estimation}
\label{Inference and Parameter Estimation}

In this section, we briefly describe how inference and parameter estimation for topic models, in particular for the STM, are performed. Inference is conducted using variational inference, where specifically a variational Expectation-Maximization (EM) algorithm is employed for empirical parameter estimation. As a detailed discussion of the underlying workings is outside the scope of this paper, we refer the reader to the appendix and the referenced papers.

Since the STM, as well as all models it builds on, are (hierarchical) Bayesian models, the central challenge we face is the exact determination of the posterior distribution. Recall that in the section above, we derived the posterior \textit{up to proportionality}, neglecting the division by marginal distributions. The exact posterior distribution is intractable to compute due to the marginal distributions in the denominator, which is why exact inference is infeasible and variational inference is used instead. Generally, for a model with latent variables $\theta$ and $z$ and observable data $x$, variational inference involves approximating the posterior $p(\theta,z|x)$ by postulating a simple distribution family $q(\theta,z)$ for the (joint) distribution of latent model variables $\theta$ and $z$ and subsequently determining the member of this family which minimizes the "distance" to the true posterior distribution, measured using the Kullback-Leibler (KL) divergence (\citealp{wang2013variational}). The approximations of variational inference bring a great amount of flexibility, but come at the cost of some bias, since the approximative distribution family usually does not contain the true posterior.

In the appendix, we show that minimizing KL divergence between true posterior $p$ and the approximating variational distribution $q$ is equivalent to maximizing a lower bound on $\log p(x)$, the log-likelihood of the observed data x. This lower bound is called \textit{ELBO} and is defined as
\begin{align*}
ELBO := \mathbb{E}_q[\log p(\theta,z,x)] - \mathbb{E}_q[\log q(\theta,z)],
\end{align*}
whose second component, $\mathbb{E}_q[\log q(\theta,z)]$, is the entropy of the approximate distribution $q$. To be precise, maximizing \textit{ELBO} (or minimizing KL divergence) refers to finding the governing parameter of the approximating distribution $q$ which maximizes \textit{ELBO}.

The optimality conditions resulting from maximizing \textit{ELBO} lead to the \textit{coordinate ascent algorithm} for variational inference (\citealp{wang2013variational}), which converges towards a local optimum (\citealp{bishop2006pattern}). However, this algorithm only works for \textit{conditionally conjugate} models, such as the LDA: all nodes in this model - in particular, the Dirichlet distribution for drawing topic proportions, the multinomial distribution for assigning topics, and the multinomial for eventually picking words - are conditionally conjugate. The STM, however, as well as the CTM before it, are non-conjugate models due to the logistic normal distribution used to sample topic proportions, which is why algorithm updates are not feasible and the algorithm is not (directly) applicable. As a remedy, \cite{wang2013variational} developed Laplace variational inference, which uses Laplace approximations within coordinate ascent algorithm updates and this way enables the application of the coordinate ascent algorithm for the broader class of nonconjugate models, in particular for CTM and STM.

As stated above, the STM uses an Expectation-Maximization (EM) algorithm for empirical parameter estimation. In the E-step, the variational posterior distributions for topic proportions and assignment, $q(\boldsymbol{\theta}_d)$ and $q(\boldsymbol{z_{d,n}})$, respectively, are updated using Laplace variational inference and coordinate ascent. In the M-step, the model parameters - specifically topical prevalence and content coefficients - are updated by maximizing \textit{ELBO} with respect to them (\citealp{roberts2016model}).
\section{Data}
\label{Data}

\subsection{Data Collection}
\label{Data Collection}

The current political landscape of Germany consists of six parties: the right-wing \textit{AfD}, the Greens (\textit{Bündnis 90/Die Grünen}), the Christian Democrats (\textit{CDU/CSU}), the Left Party (\textit{Die Linke}), the liberal \textit{FDP}, and the Social Democrats (\textit{SPD}). These parties are represented in the German parliament (\textit{Bundestag}) according to the votes obtained during the 2017 German federal election (\textit{Bundestagswahl}), which took place on September 24, 2017. The legislative period amounts to 4 years, thus ending around September 2021. The parliament currently contains a total of 709 seats. For the large majority of the 709 members of the German parliament (\textit{Abgeordnete}), information about their electoral district (\textit{Wahlkreis}) is available.

In order to analyze German political entities based on text data, we constructed a broad database containing personal and Twitter data on an MP level as well as socioeconomic and election data on an electoral-district level, as detailed in the rest of this section. While parts of this database were used in the subsequent topic model analysis, it is also to be used in future text-based analyses regarding German politics. As a first step in constructing the database, we gathered personal information on all German MPs. Using the \textit{BeautifulSoup} web scraping tool (\citealp{richardson2007beautiful}) in the Python programming language (\citealp{van1995python}) as well as a selenium webdriver, we gathered data such as name, party, biographical information, electoral district, and social media accounts from the official parliament website\footnote{https://www.bundestag.de/abgeordnete} for all of the 709 members of the German parliament during its 19th election period, elected on September 24, 2017.\footnote{As of March 30, 2020, the official parliament website contained information on 730 MPs. This is because MPs who resigned or passed away since the beginning of the election period are also listed on the website. These MPs were manually excluded from further analysis.} An additional source of personal MP-level information would be the MPs' personal homepages. However, after inspecting some of these personal homepages at random, we found that there is no systematic way to scrape them. Furthermore, hardly any of these websites contain any informative text data comparable to tweets or Facebook posts. As a consequence, we decided against further pursuing this potential source of information. Due to difficulties and recent restrictions when scraping Facebook data, caused in parts by the aforementioned data scandal, we also discarded Facebook as source of text data and focused solely on Twitter. 

Since information on social media profiles was scarce and incomplete on the official parliament website, we additionally scraped official party homepages of all of the six political parties represented in the current parliament.\footnote{The official homepage of the AfD party does not provide the Twitter profiles of their members, which is why for this party we had to manually gather the account names.} MPs who did not provide a Twitter account either on the official parliament website or on their party's official homepage were excluded. Using Python's \textit{tweepy} library (\citealp{roesslein2020tweepy}) to access the official Twitter API, we scraped all tweets by German MPs from September 24, 2017 through April 24, 2020, i.e., during a total of 31 months. This initially yielded 342,542 tweets from a total of 470 members of parliament.\footnote{\textit{tweepy} restricts the total number of tweets retrievable to 3,200. For those MPs who tweeted more than 3,200 tweets during our period of analysis, the most recent 3,200 tweets were taken into account. However, this only applied to two MPs.} The \textit{tweepy} library offers a variety of additional features to be extracted apart from the mere tweet texts, such as the number of followers of an account, retweets, or how many times a tweet was like or retweeted. While we only use original tweets in the analysis presented in this paper, we included the most relevant additional Twitter features in our database, for use in future analyses.

To complement personal and Twitter data, we also gathered socioeconomic data such as GDP per capita and unemployment rate as well as 2017 election results on an electoral-district level for all of the 299 electoral districts from the official electoral website\footnote{https://www.bundeswahlleiter.de}. After removing the only MP labeled as independent (\textit{fraktionslos}) on the official electoral website as well as 19 MPs without a specific electoral district assigned to them (for matchability with socioeconomic data), the final dataset counted 450 MPs. Overall, 63\% of all 709 MPs were thus included in the analysis. The corresponding total number of tweets amounted to 323,740. For those MPs without elecotoral district, electoral district-level socioeconomic variables could potentially be imputed by using state averages or values of nearby and/or similar districts. However, given that this only applies to 19 out of the remaining 450 MPs and since imputing covariates would introduce further uncertainty, we decided to exclude those MPs.

The table below shows total monthly tweet frequencies for our period of analysis, September 24, 2017 through April 24, 2020. As can be seen, tweet frequencies - though fluctuating - show an increasing trend over time, peaking at almost 20,000 in March 2020. The decrease for April 2020 can partly be explained by the fact that only the first 24 days of the month were taken into account.

\begin{figure}[h!]
  \centering
  \captionsetup{justification=centering,margin=2cm}
  \includegraphics[scale = 0.5]{../plots/3/monthly_tweets.pdf}
  \caption{Monthly tweet volume by German MPs from September 24, 2017 through April 24, 2020.}
  \label{fig:monthly_tweets}
\end{figure}

Next, data was grouped and tweets were concatenated on a per-user level (thus aggregating tweets across the entire 31 months) as well as on a per-user per-month level, yielding a user-level and a monthly dataset. This means that a document represents the concatenation of \textit{all} of a single MP's tweets for the user-level dataset, while it represents a single MP's \textit{monthly} tweets for the monthly dataset. This also means that MP-level metadata such as personal information and socioeconomic data (through the electoral-district matching) can be used as document-level covariates. For the monthly dataset, the temporal component (year and month) constitutes an additional covariate. Since it is reasonable to assume that the importance of topics varies over time and due to resulting documents being shorter and more easily interpretable, we chose the monthly dataset for further analysis.\footnote{For instance, as stated in section \ref{Labeling}, one topic is about COVID-19, which is clearly a relatively recent topic. The monthly dataset allows for tracing the development of this topic's relevance over time: a flat curve until January 2020, followed by a sharp increase during the first months of 2020. The user-level dataset, on the other hand, would simply assign a low overall proportion to this topic.} At this point, the data preparation was completed, marking the starting point of the preprocessing required for topic analysis, which is identical for both the user-level and the monthly dataset.

\subsection{Data Preprocessing}
\label{Data Preprocessing}

For preprocessing, we used the \textit{quanteda} package (\citealp{quanteda}) within the R programming language (\citealp{R}). As a first step, we built a quanteda corpus from all documents. Next, we immediately transcribed German umlauts \textit{ä/Ä}, \textit{ö/Ö}, \textit{ü/Ü} as well as German ligature \textit{ß} as \textit{ae/Ae}, \textit{oe/Oe}, \textit{ue/Ue}, and \textit{ss}, respectively, and removed hyphens. Subsequently, we transformed the text data into a quanteda document-feature matrix (DFM), which essentially tokenizes texts, thereby converting all characters to lowercase. From the DFM, we removed an extensive list of German stopwords, using the stopwords-iso GitHub repository\footnote{https://github.com/stopwords-iso/stopwords-iso}, as well as English stopwords included in the \textit{quanteda} package. Moreover, hashtags, usernames, quantities and units (e.g., \textit{10kg} or \textit{14.15uhr}), interjections (e.g., \textit{aaahhh} or \textit{ufff}), terms containing non-alphanumerical characters, meaningless word stumps (e.g., \textit{innen} from the German female plural declension, or \textit{amp}, the remainder left after removing the ampersand sign, \textit{\&}) were removed. Terms with less than four characters and terms with a term frequency (overall number of occurrences) below five or with a document frequency (number of documents containing the word) below three were excluded. Finally, we manually removed overly frequent terms that would diminish the distinguishability of topics, such as \textit{bundestag} or \textit{polit} (see \textit{semantic coherence} in section \ref{Hyperparameter Search and Model Fitting} for a technical explanation).

We also performed word-stemming, which means cutting off word endings to remove discrepancies arising purely from declensions or conjugations, being of particular importance for the German language. Due to the nature of the German language, the additional gains of lemmatization (which aims at identifying the base form of each word) would only be small as compared to the large increase in complexity, which is why we decided to use stemming only. Another issue when dealing with German language documents is represented by compound words, which are sometimes hyphenated, basically leading to a distinction where semantically there is none. We addressed this issue by removing hyphens in the very beginning of the preprocessing and converting all terms to lowercase, thus "gluing together" compound words; this way, terms like \textit{Bundesregierung} and \textit{Bundes-Regierung} are both transformed into \textit{bundesregierung} (and, after stemming, into \textit{bundesregier}). Finally, automatic segmentation techniques were not necesssary for the German language. An in-depth discussion of topic model preprocessing and its application to Twitter data can be found in \cite{lucas2015computer}. As a result of preprocessing, one empty MP-level document was dropped, so that a total of 10,998 (monthly) MP-level documents were eventually analyzed, each one associated with 90 covariates.
\input{./kapitel/4_global_analysis}
\section{Metadata Analysis - Topical Prevalence and Content}
\label{Metadata Analysis - Topical Prevalence and Content}

\subsection{Covariate-level Topic Analysis}
\label{Covariate-level Topic Analysis}

We now proceed to analyze the relationship between metadata information (i.e., document-level covariates) and topic proportions. We specify topical prevalence as 
\begin{align}
\mu_{d,k} = \boldsymbol{x_d}^T \boldsymbol{\gamma}_k= \gamma_{k,\text{party}_d} + \gamma_{k,\text{state}_d} + f_k(\text{t}_d) + g_k(\text{struct}_d), \label{prevalence}
\end{align} 
for all documents $d = 1,\dots,D$, and for all topics $k = 1,\dots,K$, where 
\begin{align*}
g_k(\text{struct}_d) = g_{k}^{(1)}(\text{GDP}_d)+g_{k}^{(2)}(\text{unemployment}_d)+g_{k}^{(3)}(\text{immigrants}_d)+g_{k}^{(4)}(\text{votes}_d). 
\end{align*} 
That is, the political party and federal state of the respective parliamentarian associated with a document are specified as simple categorical dummy effects, while date and electoral-district structural covariates (GDP per capita, unemployment rate, percentage of immigrants, and the 2017 vote share) are modeled as additive smooth functions.

Note that approximate inference implies replacing $\mu_{d,k}$ with $\lambda_{d,k}$, i.e., with the mean of the approximate Gaussian posterior $q(\eta_{d,k})$. The estimates of $\boldsymbol{\Gamma} = [\boldsymbol{\gamma}_1 | \dots | \boldsymbol{\gamma}_K]$ are updated in a Bayesian linear regression during each iteration of the EM algorithm in the M-step; for details see \cite{roberts2013structural}, p.\ 993.

While topical prevalence does has an effect on the estimated topic proportions, the exact specification of topical prevalence does not. Both estimated topic proportions and heldout likelihood are in general only marginally affected by the specific choice of the functional form. However, completely removing topical prevalence, in which case the model reduces to a CTM, does result in different topic proportions, as we show in section \ref{Two-step Approach: CTM}. Since evaluation metrics such as held-out likelihood are mostly unaffected by the exact choice of topical prevalence and because the computational cost of fitting an STM is rather high, automatic model selection methods with respect to topical prevalence are not available. A reasonable specification of topical prevalence therefore relies on the domain knowledge of the researcher.

There exist different approaches to study the relationship between topic proportions and prevalence covariates. One possibility is to directly assess the estimates $\hat{\boldsymbol{\Gamma}}$ and $\hat{\boldsymbol{\Sigma}}$ generated by the STM. Since the document-level topic proportions $\boldsymbol{\theta}_d$ follow a logistic normal distribution (with median $\boldsymbol{\mu}_d$ and covariance parameters $\boldsymbol{\Sigma}$), interpretation of the results can be difficult, since the logistic normal distribution is not easily accessible. Nonetheless, we can still visualize the relationship between a topic and a prevalence covariate, fixing other covariates at their median (for categorical variables the majority vote is used).

Alternatively, the estimated topic proportions can be used as dependent variable of a new regression on prevalence covariates. However, in contrast to a standard regression setting, in this case the dependent variable has been estimated itself before the regression is performed. Instead of simply using the maximum-a-posteriori (MAP) estimates of $\boldsymbol{\theta}_d$ as the dependent variable, having access to the posterior distribution of the topic proportions, we can account for the uncertainty of the dependent variable. This can be achieved by employing a sampling procedure known as the method of composition in the social sciences  (\citealp{tanner2012tools}, p.\ 52). This procedure is implemented in the \textit{stm} package through its function \textit{estimateEffect}.

In the following, we first introduce the method of composition. We discuss its implementation in the \textit{stm} package and provide alternative regression approaches based on the method of composition. Subsequently, we evaluate the relationship between prevalence covariates and topic proportions by directly assessing the estimates $\hat{\boldsymbol{\Gamma}}$ and $\hat{\boldsymbol{\Sigma}}$, as outlined above, and compare the results of the two approaches.

\subsubsection{Method of Composition}
\label{Method of Composition}

Let $\boldsymbol{\theta}_{(k)}:=(\theta_{1,k}, \dots, \theta_{D,k})^T \in [0,1]^{D}$ denote the proportions of the $k$-th topic for all $D$ documents. As stated, we want to perform a regression of these topic proportions $\boldsymbol{\theta}_{(k)}$ on a subset $\tilde{\boldsymbol{X}} \in \mathbb{R}^{D \times \tilde{P}}$ of prevalence covariates $\boldsymbol{X}$. The true topic proportions are unknown, but the STM produces an estimate of the approximate posterior of $\boldsymbol{\theta}_{(k)}$. A na{\"\i}ve approach would be to regress the estimated mode of the approximate posterior distribution on $\tilde{\boldsymbol{X}}$. However, this approach neglects much of the information contained in the distribution. 

Instead, repeatedly sampling $\boldsymbol{\theta}_{(k)}^*$ from the approximate posterior distribution of $\boldsymbol{\theta}_{(k)}$, performing a regression for each sampled $\boldsymbol{\theta}_{(k)}^*$ on $\tilde{\boldsymbol{X}}$, and then sampling from the estimated distribution of coefficients, provides an i.i.d.\ sample from the marginal posterior distribution of regression coefficients. 

Sampling $\boldsymbol{\theta}_{(k)}^*$ is achieved by first sampling the unnormalized topic proportions $\boldsymbol{\eta}^*$ from the approximate posterior $q(\boldsymbol{\eta})$, applying the softmax $\boldsymbol{\theta}^* = \text{softmax}(\boldsymbol{\eta}^*)$ (element-wise, i.e., for each of the K-dimensional vectors of topic proportions), and lastly selecting the $k$-th column of $\boldsymbol{\theta}^*$. Precisely, $q(\boldsymbol{\eta}) = \prod_d q(\boldsymbol{\eta}_d)$ is a normal distribution, which emerges from the Laplace approximation within variational inference; for details see \cite{roberts2016model}, pp.\ 992-993. For clarity, we denote the approximate posterior of topic proportions as $q(\boldsymbol{\theta}_{(k)} | \boldsymbol{X}, \boldsymbol{W})$, in order to emphasize that the parameters of this distribution are learned from the observed data, i.e.\ prevalence covariates $\boldsymbol{X}$ and words $\boldsymbol{W}$ (note that we have not included any content variables $\boldsymbol{Y}$). Furthermore, let $\boldsymbol{\xi}$ denote the regression coefficients from a regression of $\boldsymbol{\theta}_{(k)}$ on $\tilde{\boldsymbol{X}}$, and let $q(\boldsymbol{\xi} | \tilde{\boldsymbol{X}}, \boldsymbol{\theta}_{(k)})$ be the approximate posterior distribution of these coefficients, i.e.\ given design matrix $\tilde{\boldsymbol{X}}$ and response $\boldsymbol{\theta}_{(k)}$.

The method of composition can now be described by repeating the following process $m$ times:
\begin{enumerate}
\item Draw $\boldsymbol{\theta}_{(k)}^* \sim q(\boldsymbol{\theta}_{(k)} | \boldsymbol{X}, \boldsymbol{W})$.
\item Draw $\boldsymbol{\xi}^* \sim q(\boldsymbol{\xi} | \tilde{\boldsymbol{X}}, \boldsymbol{\theta}_{(k)})$.
\end{enumerate}
It then holds that $\boldsymbol{\xi}_1^*, \dots, \boldsymbol{\xi}_m^*$ is an i.i.d.\ sample from the marginal posterior
\begin{align*}
q(\boldsymbol{\xi} | \boldsymbol{X}, \boldsymbol{W}) := \int_{\boldsymbol{\theta}_{(k)}} q(\boldsymbol{\xi} | \tilde{\boldsymbol{X}}, \boldsymbol{\theta}_{(k)}) q(\boldsymbol{\theta}_{(k)} | \boldsymbol{X}, \boldsymbol{W}) \text{d} \boldsymbol{\theta}_{(k)} = \int_{\boldsymbol{\theta}_{(k)}} q(\boldsymbol{\xi}, \boldsymbol{\theta}_{(k)} | \boldsymbol{X}, \boldsymbol{W}) \text{d} \boldsymbol{\theta}_{(k)}, 
\end{align*}
where $q(\boldsymbol{\xi}, \boldsymbol{\theta}_{(k)} | \boldsymbol{X}, \boldsymbol{W}) := q(\boldsymbol{\xi}| \tilde{\boldsymbol{X}}, \boldsymbol{\theta}_{(k)}) q(\boldsymbol{\theta}_{(k)} | \boldsymbol{X}, \boldsymbol{W})$. Thus, by integrating over $\boldsymbol{\theta}_{(k)}$, this approach allows incorporating information contained in the posterior distribution of $\boldsymbol{\theta}_{(k)}$ when determining $\boldsymbol{\xi}$.\\
\\
\textbf{Implementation in the \textit{stm} package} \vspace{10px}

\noindent The R package \textit{stm} implements a simple OLS regression through its \textit{estimateEffect} function. However, this approach ignores that the sampled topic proportions are restricted to $(0,1)$ by construction. As expected, using this framework we frequently observe predicted proportions outside of $(0,1)$. Moreover, credible intervals are non-informative, due to violated model assumptions. \\

\begin{figure}[h!]
  \centering
  \captionsetup{justification=centering,margin=2cm}
  \begin{subfigure}[b]{0.4\linewidth}
    \includegraphics[width=\linewidth]{../plots/5_1/estEffect_topic1.pdf}
  \end{subfigure}
  \begin{subfigure}[b]{0.4\linewidth}
    \includegraphics[width=\linewidth]{../plots/5_1/estEffect_topic6.pdf}
  \end{subfigure}
  \caption{Estimated prevalence of topics 1 and 6 over time, generated using \textit{estimateEffect} from the \textit{stm} package}
  \label{fig:estEffect_topic16}
\end{figure}

\noindent \textbf{Alternative implementation} \vspace{10px}

\noindent We can attempt to improve the approach employed within the \textit{stm} package by replacing the OLS regression with a regression model that assumes a dependent variable in the interval $(0,1)$. However, note that distributional assumptions about topic proportions $\boldsymbol{\theta}_{(k)}$ will still be violated, unless one specifies the $K$-dimensional vector of topic proportions as the response variable of a multivariate regression. Otherwise inter-topic correlations are neglected - regardless of the specific univariate regression model employed. This is due to the fact that the distribution of a subvector - and thus particularly of a single component - of $\boldsymbol{\theta}_d$ is not of a simple form when $\boldsymbol{\theta}_d$ follows a logistic normal distribution, see e.g.\ \cite{atchison1980logistic}.

As shown by \cite{atchison1980logistic}, a distribution that can be used to approximate a logistic normal distribution is the Dirichlet distribution. However, note that the Dirichlet distribution assumes less interdependence among components than implied by the logistic normal distribution, as discussed in section \ref{Theoretical Framework}. In case of the Dirichlet distribution the univariate marginal distributions are beta distributions. One possibility is thus to perform a separate beta regression for each topic proportion on $\tilde{\boldsymbol{X}}$. 

As an alternative approximation we can employ a quasibinomial generalized linear model (GLM). Topic proportions can be rescaled and discretized and topics understood as classes, such that each rescaled topic proportion can be interpreted as the "number of successes" for the respective class. To match the underlying logistic normal distribution more closely, the quasi-likelihood furthermore allows for a flexible variance specification.

Note that $q(\boldsymbol{\xi}| \boldsymbol{\theta}_{(k)}, \tilde{\boldsymbol{X}})$ is asymptotically normal for both the beta regression (\citealp{ferrari2004beta}, p.\ 17) and the quasibinomial GLM (\citealp{fahrmeir2007regression}, p. 285). In both cases, we use a logit-link. \\
\\
\noindent \textbf{Visualization} \vspace{10px}

\noindent We now apply the method of composition, based on either a beta regression or a quasibinomial GLM, in order to visualize covariate effects. Here we only visualize the results obtained by the quasibinomial GLM; the results of the beta regression, which show similar trends, are found in the appendix. Setting the number of simulations to 100, we sample $\boldsymbol{\xi}^*_1, \dots, \boldsymbol{\xi}^*_{100}$ from the  marginal posterior distribution $q(\boldsymbol{\xi} | \boldsymbol{X}, \boldsymbol{W})$. As mentioned, when visualizing the impact of a particular covariate, all other covariates are held at their median (or majority vote, if categorical), in line with the methodology employed in the \textit{stm} package.
Let $\tilde{\boldsymbol{X}}^*$ denote the subset of $\boldsymbol{X}$ where, apart from the variable of interest, each selected column consists of the median of the respective column of $\boldsymbol{X}$. In order to plot the predicted effects, we then input $\tilde{\boldsymbol{X}^*}\boldsymbol{\xi}^*$ into the sigmoid function, which is the response function corresponding to a regression with logit-link, and calculate the predicted proportions. 

We exemplarily illustrate the relationship between covariates and topic proportions for topic 4 ("Social/Housing") and topic 6 ("Climate Protection"). The linear predictor of our regressions takes the same form as in  \eqref{prevalence}, i.e., we do not use a subset $\tilde{\boldsymbol{X}}$ but the full set of prevalence covariates $\boldsymbol{X}$ in order to estimate the effects, although we do not display each covariate included. For smooth effects, it is important to recall that their borders are inherently unstable, which is why one should refrain from (over-)interpreting them. For both continuous and categorical variables, black lines indicate the mean and the shaded area represents 95\% credible intervals.

\begin{figure}[h!]
  \centering
  \captionsetup{justification=centering,margin=2cm}
  \begin{subfigure}[b]{0.49\linewidth}
    \includegraphics[width=\linewidth]{../plots/5_1/quasi_t4_cont.pdf}
  \end{subfigure}
  \begin{subfigure}[b]{0.49\linewidth}
    \includegraphics[width=\linewidth]{../plots/5_1/quasi_t6_cont.pdf}
  \end{subfigure}
  \caption{Mean and 95\% credible intervals for smooth effects, obtained using a quasibinomial GLM.}
  \label{fig:quasi_t46_cont}
\end{figure}

\begin{figure}[h!]
  \centering
  \captionsetup{justification=centering,margin=2cm}
  \begin{subfigure}[b]{0.49\linewidth}
    \includegraphics[width=\linewidth]{../plots/5_1/quasi_t4_cat.pdf}
  \end{subfigure}
  \begin{subfigure}[b]{0.49\linewidth}
    \includegraphics[width=\linewidth]{../plots/5_1/quasi_t6_cat.pdf}
  \end{subfigure}
  \caption{Mean and 95\% credible intervals for different political parties, obtained using a quasibinomial GLM.}
  \label{fig:quasi_t46_cat}
\end{figure}

For topic 4, "Social/Housing", we observe that most continuous variables have a small effect in absolute terms: the absolute variation in topic proportion across the covariate domains merely amounts to 4\%, compared to 8\% for topic 6. For most covariates the trend is rather ambiguous. Somewhat surprisingly, a very high unemployment rate is negatively linked to topic 4.

The effect of the political party on the relevance assigned to the topic "Social/Housing" is very much in line with a priori expectations: the left party and social democrats have the highest topical prevalence (15\% and 10\%, respectively), and the nationalist party the lowest (2\%).

For the smooth effects of topic 6, we observe its prevalence peaks in September 2019, corresponding to month $t=25$, decreasing afterwards. The absolute changes in topic proportions over time are rather small (around 3\%). The percentage of immigrants within an electoral district shows a negative relation with topic 6. Furthermore, topic 6 tends to be discussed more frequently in mid-income electoral districts than in high- or low-income districts. Finally, the link to the unemployment rate is somewhat ambiguous, although generally positive.

Regarding the relationship between the political party and the prevalence of topic "Climate Protection", we find high topical prevalence for the green party, as expected. Similar to the smooth effects, total variation in topic proportions across parties amounts to approximately 8\%.

Finally, the graph below shows a summary comparison of topical prevalence across all parties, for topics "Right/Nationalist", "Climate Protection" and "Social/Housing". The results are generally consistent with expectations. The proportions of topics "Climate Protection" and "Social/Housing" vary between 2\% and 9\% and between 2\% and 15\%, respectively. For topic 1, "Right/Nationalist", note how topical prevalence for the AfD party amounts to more than 40\%, implying that more than 40\% of the total content tweeted by AfD party members is about right-wing/nationalist issues, particularly immigration; for all other parties, topic 1 is rather marginal below 3\%.

\begin{figure}[h!]
  \centering
  \captionsetup{justification=centering,margin=2cm}
  \includegraphics[scale = 0.5]{../plots/5_1/quasi_t146_cat.pdf}
  \caption{Topical prevalence by political party for topics 1, 4, and 6.}
  \label{fig:quasi_t146_cat}
\end{figure}

\subsubsection{Direct assessment using $\hat{\Gamma}$ and $\hat{\Sigma}$}
\label{Direct assessment}

The STM being an extension of the correlated topic model (CTM), it is assumed that the topic proportions follow a logistic normal distribution, such that $\boldsymbol{\theta}_d \sim \text{LogisticNormal}_{K-1}(\boldsymbol{\Gamma}^T\boldsymbol{x}_d^T, \boldsymbol{\Sigma})$. Within the CTM, the Dirichlet distribution of the LDA has been replaced with a logistic normal distribution, in order to allow for a joint dependence among topics. Therefore, as mentioned above, separately modeling topic proportions is a simplification; in particular credible intervals should be treated with caution.

In order to examine the relation of prevalence covariates and topic proportions considering the joint dependence among latter ones, we can attempt to directly use the output produced by the STM: inference within the STM involves finding the maximum-a-posteriori (MAP) estimate $\hat{\boldsymbol{\Gamma}}$ and the maximum likelihood estimate $\hat{\boldsymbol{\Sigma}}$. 

If we are interested in how a specific prevalence variable is related to topic proportions, similarly to previous analyses we can attempt to predict topic proportions based on a new design matrix $\boldsymbol{X}^*$, where each column, apart from the variable of interest, corresponds to the median of the respective column of $\boldsymbol{X}$. Ideally, in order to directly predict topic proportions, we would first draw a sample $\boldsymbol{\Gamma}^*$ from the posterior distribution of $\boldsymbol{\Gamma}$ and subsequently sample the topic proportions $\boldsymbol{\theta}_d^*$ from a logistic normal with mean parameters $((\boldsymbol{\Gamma}^*)^T (\boldsymbol{x}_d^*)^T, \boldsymbol{\Sigma}^*)$, where $\boldsymbol{\Sigma}^*$ is the maximum likelihood estimation of $\boldsymbol{\Sigma}$ updated according to the sampled $\boldsymbol{\Gamma}^*$ (see \citealp{roberts2016model}, p.\ 993, formula 11). The resulting topic proportions would then correspond to a sample of the posterior predictive distribution of topic proportions. Unfortunately, the output of the STM does not allow for the possibility to draw a sample from the posterior distribution of $\boldsymbol{\Gamma}$, but only provides its MAP estimate $\hat{\boldsymbol{\Gamma}}$. 

Nevertheless, in order to get an impression of how the assumed generative process of topic proportions in the STM behaves, we can plug the estimates $\hat{\boldsymbol{\Gamma}}$ and $\hat{\boldsymbol{\Sigma}}$ into the logistic normal distribution and visualize sampled values from this distribution. Given a new observation $\boldsymbol{x}_d^*$, we can sample $\boldsymbol{\theta}_d^*$ from $\text{LogisticNormal}_{K-1}(\hat{\boldsymbol{\Gamma}}^T(\boldsymbol{x}_d^*)^T, \hat{\boldsymbol{\Sigma}})$ by:

\begin{enumerate}
\item Drawing $\boldsymbol{\eta}_d^* \sim \mathcal{N}_{K-1}(\hat{\boldsymbol{\Gamma}}^T(\boldsymbol{x}_d^*)^T, \hat{\boldsymbol{\Sigma}})$ and setting $\boldsymbol{\eta}^*_{d,K} = 0$.
\item Mapping to the simplex, i.e., for all $k = 1,\dots,K$: $\theta_{d,k}^* = \frac{\exp(\eta^*_{d,k})}{\exp(\sum_{i=1}^{K} \eta^*_{d,i})}$.
\item Setting $\boldsymbol{\theta}_d^* := (\theta_{d,1}^*, \dots \theta_{d,K}^*)^T$.
\end{enumerate}

We repeated the above steps 1000 times for each input value of a selected variable, while fixing other variables at their median, and obtained the empirical mean as well as 95\% credible intervals. Plotting the results, we observe that while the mean shows a similar trend to our previous analyses, the obtained credible intervals are much broader.

\begin{figure}[h!]
    \centering
     \captionsetup{justification=centering,margin=2cm}
  \begin{subfigure}[b]{0.3\linewidth}
    \includegraphics[width=\linewidth]{../plots/5_1/direct_t6_without_credible.pdf}
  \end{subfigure}
  \begin{subfigure}[b]{0.3\linewidth}
    \includegraphics[width=\linewidth]{../plots/5_1/direct_t6_with_credible.pdf}
  \end{subfigure}
  \begin{subfigure}[b]{0.3\linewidth}
    \includegraphics[width=\linewidth]{../plots/5_1/direct_t6_cat.pdf}
  \end{subfigure}
  \caption{Smooth effects without credible intervals (left), smooth effects with credible intervals (mid), and effect of the political party (right).}
  \label{fig:directassessment}
\end{figure}

The high uncertainty that is observed can be ascribed to the fact that the unnormalized topic proportions are drawn from a $K-1$-dimensional \textit{multivariate} normal distribution before the softmax is applied. Therefore, a single normalized proportion depends heavily on the sampled unnormalized proportions of the remaining topics. While the variance of a topic-specific unnormalized proportion is independent of the remaining unnormalized proportions and c.p.\ constant for an increasing number of topics, the application of the softmax function induces a large increase in the variance of a topic-specific normalized proportion.

We suspect that the magnitudes of credible intervals in figure \ref{fig:directassessment} provide a more realistic picture than those of a separate modeling of topic proportions, since the usage of the logistic normal distribution of topic proportions means implicitly assuming dependence among topics, as argued above. This ultimately produces a large variance of the univariate marginal distributions of topic proportions, as can be observed. Ideally, one should sample $\boldsymbol{\Gamma}$ from its posterior distribution instead of plugging in its MAP estimate. Still, our results suggest that there is a discrepancy between the distribution of topic proportions assumed in the generative process of the STM and the impression we gain of this distribution via a separate modeling of topics using the method of composition.


\subsection{Topical Content}
\label{Topical Content}

The STM provides an additional way to integrate covariate effects into the model, apart from prevalence variables that impact topic proportions across documents. To be specific, a categorical variable can be selected as topical content variable. While the prevalence variables influence the propensity of the 15 topics for each document, the content variable now allows for the word distributions for a given topic to vary across documents, according to the content variable level. Note that this is a completely new model, which is why one should not expect the resulting topics to be similar. 

Formally, recall that the word distribution used to eventually pick a word is $\boldsymbol{\beta}_{d,n} := \boldsymbol{\beta}(\boldsymbol{z_{d,n}}, Y_d) \in \mathbb{R}^V$, where $\boldsymbol{z_{d,n}}$ is a (latent) indicator variable determining the word's topic assignation and $Y_{d}$ is the document-level topical content variable with $A$ levels. In the prevalence model, no (document-level) topical content variable is specified, implying $\boldsymbol{\beta}_{d,n} = \boldsymbol{\beta}(\boldsymbol{z_{d,n}})$; since $\boldsymbol{z_{d,n}}$ is a word-level variable, $\boldsymbol{\beta}_{d,n}$ is constant across all documents for a given topic $k$. When specifying a content variable $\boldsymbol{Y}$, however, $\boldsymbol{\beta}_{d,n}$ now varies for each document, according to the level $a \in \{1, ..., A\}$ the content variable takes on for document $d$. That is, the total number of $\boldsymbol{\beta}$-vectors, each one of length $V$, now increases from $K$ to $K \cdot A$.

For our specific case, since the topical content variable needs to be categorical, we choose the variable \textit{party}, being categorical by definition. In doing so, we implicitly posit that for a given topic, an MP's party additionally influences the vocabulary used when tweeting about that specific topic. For instance, this implies that an AfD party member tweets about immigration issues in a different linguistic manner than, say, a green MP. Since for the 2017 election period the German parliament contains members of six parties, $\boldsymbol{Y}$ is now technically a matrix with 10998 rows and six columns,\footnote{Recall that the dimensions of matrix $\boldsymbol{Y}$ are due to dummy encoding of the topical content variable in the \textit{stm} model implementation, whereas for notational simplicity we refer to $\boldsymbol{Y}$ as vector and $Y_d$ as scalar.} yielding a total of $15 \cdot 6 = 90$ $\boldsymbol{\beta}$-vectors.

After fitting the model, we proceed as for the prevelance model, that is, by inspecting top words and identifying topic labels. An additional difficulty, however, is that we do not have clear-cut top words per topic anymore; instead, we now have topic-level top words for each of the 15 topics, party-level top words for each of the six parties, as well as interaction top words for each of the 90 topic-party combinations. The table below presents topic labels for all 15 topics, identified by using the same three-step procedure as for the prevalence model before. As can be seen, five topics are labeled as \textit{miscellaneous}, reflecting the complexity caused by the large number of $\boldsymbol{\beta}$-vectors.

\begin{table}[h!]
	\centering
	\captionsetup{justification=centering,margin=2cm}
	\begin{tabular}{|l|l|}
	\hline
	Topic1  & Right/Nationalist 1  \\ \hline
	Topic2  & Miscellaneous 1      \\ \hline
	Topic3  & Left/Humanitarian    \\ \hline
	Topic4  & Housing       	   \\ \hline
	Topic5  & Innovation           \\ \hline
	Topic6  & Green/Energy         \\ \hline
	Topic7  & Miscellaneous 2      \\ \hline
	Topic8  & Corona               \\ \hline
	Topic9  & Foreign Affairs      \\ \hline
	Topic10 & Election             \\ \hline
	Topic11 & Right/Nationalist 2  \\ \hline
	Topic12 & Miscellaneous 3      \\ \hline
	Topic13 & Miscellaneous 4      \\ \hline
	Topic14 & Twitter/Politics     \\ \hline
	Topic15 & Miscellaneous 5      \\ \hline
	\end{tabular}
	\caption{List of topic labels for STM with topical content variable (party).}
	\label{Tab:labels_content}
\end{table}

The topical content model allows for vocabulary usage to differ across political parties, given a topic. In Figure \ref{fig:t8_vocab_parties} below, we visualize this effect for the Corona topic, contrasting the green party "Bündnis 90/Die Grünen" with the right-wing nationalist party "AfD". The result is very insightful: even for a topic as clear-cut and novel as COVID-19, stark differences in terms of vocabulary usage arise. In particular, the AfD uses language suitable to describe immigration (\textit{migration}, \textit{grenz}) in order to discuss Corona, which very much reflects the unimodality of the party's political orientation (as can also be seen in Figure \ref{fig:quasi_t146_cat} at the end of section \ref{Method of Composition}). The green party, on the other hand, seems to address the topic much more specifically, mentioning key words like \textit{massnahm} or \textit{kind}.

\begin{figure}[h!]
  \centering
  \captionsetup{justification=centering,margin=2cm}
  \includegraphics[scale = 0.5]{../plots/5_2/t8_vocab_parties.pdf}
  \caption{Differences in vocabulary usage across parties for the Corona topic.}
  \label{fig:t8_vocab_parties}
\end{figure}

While this type of visualization is indeed insightful, several concerns regarding the topical content model prevail: first of all, there is no natural candidate for the content variable, which - for labeling and interpretational purposes - should ideally be binary. Our dataset contains very few categorical variables, none of them binary. Furthermore, there is no natural, non-arbitrary way to binarize any of the covariates; for instance, binarizing the variable party into conservative and liberal would misclassify at least one party. Therefore, our choice to use party as content variable is the result of a lack of alternatives, rather than being based on sound statistical or theoretical considerations. This, in turn, is reflected in the difficulties with the labeling: recall that one third of all topics were eventually being labeled as miscellaneous. And while the previous illustration of inter-party differences in vocabulary usage is indeed insightful in terms of topic exploration and visualization, the aforementioned doubts lead us to discard the topical content variable for further analysis. In fact, in the next section we consider a model without any metadata covariates.
\input{./kapitel/6_1_ctm}
\subsection{Causal Inference: Train-test Split}
\label{Causal Inference: Train-test Split}

In section \ref{Covariate-level Topic Analysis}, we first estimated latent topic proportions using the STM and then assessed the relation between these document-level topic proportions and prevalence covariates. In particular, the documents that were used to obtain the topic proportions were the same that were subsequently used to quantify relationships between covariates and topic proportions. As \cite{egami2018make} argue, this double usage of data causes both an identification problem and an overfitting problem; hence inferences about covariate effects are biased. Additionally, since in the STM prevalence covariates affect estimated topic proportions, there is not only a double usage of data (i.e., in the sense that the same documents are used twice), but also a direct double usage of prevalence covariates as the estimated latent topic proportions are regressed on the former.

The problems arising from double usage of documents are best understood when considering a classical causal inference scenario. Therefore, assume that there are two groups, a treatment group and a control group. Aside from treatment, individuals from both groups are similar. The objective is to quantify the treatment effect, which in our case is the effect of treatment on the prevalence of a specific topic. The identification problem occurs because estimating the topic model in order to discover latent topic proportions can introduce an additional dependency among individuals. In such a case the response of an individual (i.e., the topic proportion) depends not only on treatment of this individual, but also on treatment of other individuals. Consequently, the estimation of treatment effects in the second stage is biased, since the assumption that the response is only determined by treatment of that individual is not valid. In addition to this identification problem, we might face an instance of overfitting: as with any model, we likely mistake noise for patterns. In such a case, again, the response is not solely determined by treatment of an individual, but additionally by specific characteristics of other individuals. This also results in a biased estimate of the treatment effect.

The problems described can be addressed using a framework proposed by \cite{egami2018make}. The general idea is to split the data $\mathcal{D}$ into a training set $\mathcal{D}_{\text{train}}$ and a test set $\mathcal{D}_{\text{test}}$. The training set is used to determine a model in order to infer latent topic proportions from any text assumed to be generated by the same underlying process as the training set. Subsequently, this estimated model is applied to the test set in order to assess the relation between test set topic proportions and test set prevalence covariates. The train-test split solves the identification problem, because when predicting the topic proportion for an individual on the test set, there is no dependence on treatment of other individuals from the test set since the model used for prediction is determined by training set observations only. Therefore, when estimating the treatment effect by comparing predicted proportions between different test set observations, the assumption that each proportion is independent of other \textit{test set} individuals' treatment holds. Likewise, idiosyncratic noise from the training set, which determines the model used to predict test set topic proportions, is unlikely to be found on the test set. Thus, the problem of overfitting is also solved. In the following, we explain the exact procedure for the STM (note that \cite{egami2018make} focus, for the most part, on the general framework, while the technical details of the implementation within the STM are not discussed in depth) and evaluate the results when applied to our data.

\subsubsection{Model Estimation on the Training Set}
\label{Model Estimation on the Training Set}

On the training set, we estimate components of the STM similarly to the estimation on the full data set. That is, we input documents, i.e., words and metadata from the training set, and obtain estimates $(\hat{\boldsymbol{\beta}}_{\text{train}}, \hat{\boldsymbol{\Gamma}}_{\text{train}}, \hat{\boldsymbol{\Sigma}}_{\text{train}})$, where $\hat{\boldsymbol{\beta}}_{\text{train}}$ is associated with the topic-word distribution and $\hat{\boldsymbol{\Gamma}}_{\text{train}}$ as well as $\hat{\boldsymbol{\Sigma}}_{\text{train}}$ are the topical prevalence parameters. 

\subsubsection{Prediction of Topic Proportions on the Test Set}
\label{Prediction of Topic Proportions on the Test Set}

Prediction of topic proportions on the test is not straightforward, since the topic proportions are latent and the STM was not designed for the specific purpose of predicting these latent variables on a set of new, unseen data. The fundamental idea is to estimate the variational posterior of the latent variables, that is, the topic proportions $\boldsymbol{\theta}_d$, where $d \in \mathcal{D}_{\text{test}}$ (note that $\boldsymbol{z}_d$ is integrated out in the STM), conditional on the model parameters $(\hat{\boldsymbol{\beta}}_{\text{train}}, \hat{\boldsymbol{\Gamma}}_{\text{train}}, \hat{\boldsymbol{\Sigma}}_{\text{train}})$ from the training set as well as the words $\boldsymbol{W}_{\text{test}}$ from the test set. This functionality is implemented in the \textit{stm} package through the function \textit{fitNewDocuments}, which by default outputs the MAP estimates of topic proportions $\boldsymbol{\theta}_d$, for all $d \in \mathcal{D}_{\text{test}}$. Note that estimating the variational posterior of the latent variables, conditioned on the parameters and the words, is precisesly what occurs during each E-step of the EM Algorithm. Thus, the implementation of \textit{fitNewDocuments} simply consists of one E-step with inputs $(\hat{\boldsymbol{\beta}}_{\text{train}}, \hat{\boldsymbol{\Gamma}}_{\text{train}}, \hat{\boldsymbol{\Sigma}}_{\text{train}}, \boldsymbol{W}_{\text{test}})$. It is, however, not obvious how to exactly input $\hat{\boldsymbol{\Gamma}}_{\text{train}}$ and  $\hat{\boldsymbol{\Sigma}}_{\text{train}}$ into the E-step. Depending on the characteristics of the specific analysis conducted by the researcher, \cite{egami2018make} propose three different alternatives:
\begin{enumerate}
\item \textbf{Covariate-specific prior}: Before applying the E-step, $\hat{\boldsymbol{\Gamma}}_{\text{train}}$ is used to obtain $\hat{\boldsymbol{\mu}}_d := (\hat{\boldsymbol{\Gamma}}_{\text{train}})^T(\boldsymbol{x}_d)^T$, for each document $d \in \mathcal{D}_{\text{test}}$ in the test set. Each document is then updated performing the E-step with inputs $(\boldsymbol{\mu}_d, \boldsymbol{\Sigma}, \boldsymbol{\beta}) = (\hat{\boldsymbol{\mu}}_d, \hat{\boldsymbol{\Sigma}}_{\text{train}}, \hat{\boldsymbol{\beta}}_{\text{train}})$ together with the respective document-specific words; for the exact update machanism see \cite{roberts2013structural}, pp. 992-993. The problem with this approach is, however, that for two documents from the test set containaing the exact same words, different topic proportions are predicted if the prevalence covariates differ. However, in such a case we would want the causal effect of the covariates on the topic proportions to be zero.
\item \textbf{Average prior}: The average prior circumvents the problem of the covariate-specific prior, as described above, by simply using - for each document in the test set - the average $\overline{\boldsymbol{\mu}}_{\text{train}} := \frac{1}{|\mathcal{D}_{\text{train}}|}\sum_{d \in \mathcal{D}_{\text{train}}} (\hat{\boldsymbol{\Gamma}}_{\text{train}})^T(\boldsymbol{x}_d)^T$ of all document-specific means from the training set. The covariance $\hat{\boldsymbol{\Sigma}}_{\text{train}}$ - which we now denote as $\overline{\boldsymbol{\Sigma}}_{\text{train}}$ - is recalculated based on the new average $\overline{\boldsymbol{\mu}}_{\text{train}}$ according to formula (11) on p.\ 993 in \cite{roberts2013structural}. The E-step for each document $d \in \mathcal{D}_{\text{test}}$ from the test set is accordingly performed with inputs $(\boldsymbol{\mu}_d, \boldsymbol{\Sigma}, \boldsymbol{\beta}) = (\overline{\boldsymbol{\mu}}_{\text{train}}, \overline{\boldsymbol{\Sigma}}_{\text{train}}, \hat{\boldsymbol{\beta}}_{\text{train}})$ together with the document-specific words. In this scenario, prevalence covariates from the test set have no influence at all on the prediction of test set topic proportions. 
\item \textbf{No prior}: If no prior is used, then for each document $d \in \mathcal{D}_{\text{test}}$ in the test set the E-step is performed using $\boldsymbol{\mu}_d=0$ and replacing $\hat{\boldsymbol{\Sigma}}_{\text{train}}$ with a diagonal covariance matrix with very large diagonals.
\end{enumerate}
The covariate-specific prior cannot be used in our case due to the problem described above, that is, different topic proportions being predicted for identically worded test set documents if their prevalence covariates differ. The option "no prior" can be useful if the metadata on the test set is believed to be linked differently to topics than is the case on the training set. In most cases the second option, "average prior", should provide the best trade-off, since in this case metadata from the training set is directly used to predict topic proportions, but the problem of the covariate-specific prior is solved. Note that consequently there is no double usage of covariates in this case.

\subsubsection{Estimation of the Average Treatment Effect}
\label{Estimation of the Average Treatment Effect}

Following \cite{egami2018make}, we define the Average Treatment Effect (ATE) on prevalence of topic $k$ as
\begin{align}
\text{ATE}_k := \mathbb{E}[\theta^{\text{[treatment]}}_{d,k} - \theta^{\text{[control]}}_{d,k}],
\end{align}
where $\theta^{\text{[treatment]}}_{d,k}$ and $\theta^{\text{[control]}}_{d,k}$ denote the topic proportions for the $d$-th document and $k$-th topic under treatment and control, respectively. (Note that for document $d$, we only observe \textit{either} $\theta^{\text{[treatment]}}_{d,k}$ \textit{or} $\theta^{\text{[control]}}_{d,k}$.) That is, we are interested in the average effect of treatment on topic proportion $k$ of an individual, assuming that this average effect is identical across all individuals. In other words, we assume the change in a topic proportion induced by treatment is a random variable with equal mean for all individuals. 

In order to estimate $\text{ATE}_k$, as reasoned above, it is crucial to separate the documents used for constructing the mechanism to discover latent topic proportions from the documents to which we apply this mechanism. Formally, using either the option "no prior" or "average prior", we can denote this mechanism as a function $g_{\text{train}}$, which we determine on the training data together with the parameters $(\hat{\boldsymbol{\beta}}_{\text{train}}, \hat{\boldsymbol{\Gamma}}_{\text{train}}, \hat{\boldsymbol{\Sigma}}_{\text{train}})$. The prediction of the $k$-th topic proportion for a test set observation $d \in \mathcal{D}_{\text{test}}$ (as outlined above) can then be written as $\hat{\theta}_{d,k} = g_{\text{train}}(\boldsymbol{w_d}, \hat{\boldsymbol{\beta}}_{\text{train}}, \hat{\boldsymbol{\Gamma}}_{\text{train}}, \hat{\boldsymbol{\Sigma}}_{\text{train}})$, where $\boldsymbol{w_d}$ denotes the observed words of document $d$. 

To estimate the treatment effect on a data set $\mathcal{D}$, we determine the average difference of predicted topic proportions between both groups, i.e.,
\begin{align}
\widehat{\text{ATE}_k} = \frac{1}{|\mathcal{D}_{\text{treatment}}|}\sum_{d \in \mathcal{D}_{\text{treatment}}} \hat{\theta}_{d,k} - \frac{1}{|\mathcal{D}_{\text{control}}|}\sum_{d \in \mathcal{D}_{\text{control}}} \hat{\theta}_{d,k},
\end{align} 
where $\hat{\theta}_{d,k}$ is the predicted topic proportion for the $d$-th document and $k$-th topic. \cite{egami2018make} show that, if additional conditions hold, $\widehat{\text{ATE}_k}$ estimated on previously unseen test data $\mathcal{D}_{\text{test}}$ is an unbiased estimate of $\text{ATE}_k$.\footnote{Precisely, we predict $\hat{\theta}_{d,k} = g_{\text{train}}(\boldsymbol{w_d}, \hat{\boldsymbol{\beta}}_{\text{train}}, \hat{\boldsymbol{\Gamma}}_{\text{train}}, \hat{\boldsymbol{\Sigma}}_{\text{train}})$, for each document $d \in \mathcal{D}_{\text{test}}$.}
In contrast, if we do not split the data and "na{\"i}vely" predict topic proportions on the same data used to estimate the topic model, we obtain a biased estimate, due to both the identification problem and the overfitting problem described above.

\subsubsection{Results}
\label{Results}

We now depict our results from the train-test split, where we split the data into two equally sized sets, for the options "average prior" and "no prior". Note that the test data cannot consist of words which have not been seen in the training data. Therefore, all previously unseen words are removed from the test data. After removing the words, the test data contains 80.6\% of the original words. Since we use only a subset of the full data, the estimated topics are slightly different from those obtained using the full data; however, most topics are similar. We assign new labels to the topics, a complete list of which can be found in the accompanying R code of this paper.

In contrast to section \ref{Covariate-level Topic Analysis}, the focus of this section lies on quantifying causal effects between covariates and the relevance of a topic, since the train-test framework is most appropriate to conduct such types of analyses. As mentioned before, \textit{fitNewDocuments} outputs the MAP estimates of the variational posterior of topic proportions for the test set. In Figure \ref{fig:causal_inference_props} we depict these MAP estimates of topic proportions, along with topic proportions obtained for the training data, for two selected topics.

The UN Climate Action Summit 2019 was held on September 23, 2019. As can be observed in the left panel of Figure \ref{fig:causal_inference_props} below, the topic associated with climate issues was discussed to a much larger extent during that time than the year before. While the MAP estimates for the different prior specifications on the test set are rather similar, the estimated effect for the training data is much larger. If we compare the estimated topic proportions for a topic we labelled as 'Emancipation' for the two opposing parties 'AfD' and 'B{\"u}ndnis 90/Die Gr{\"u}nen', we find similar results (see right panel of Figure \ref{fig:causal_inference_props}): the average difference of estimated topic proportions between both parties is larger on the training data. Also, note that credible intervals on the training data differ from those on the test data in both cases.

\begin{figure}[h!]
  \centering
  \captionsetup{justification=centering}
  \begin{subfigure}[b]{0.49\linewidth}
    \includegraphics[width=\linewidth]{../plots/6_2/climate_summit_props.pdf}
  \end{subfigure}
  \begin{subfigure}[b]{0.49\linewidth}
    \includegraphics[width=\linewidth]{../plots/6_2/emancipation_props.pdf}
  \end{subfigure}
  \caption{Maximum-a-posteriori (MAP) estimates of topic proportions on training and test data. Points display the mean, lines 2.5\% and 97.5\% credible intervals.}
  \label{fig:causal_inference_props}
\end{figure}

In Figure \ref{fig:causal_inference_ate} we visualize the ATE estimated on training and test data with different prior specifications (note that this is simply the difference of the means depicted in Figure \ref{fig:causal_inference_props}). The results confirm that there is a substantial difference between the estimated effects.

\begin{figure}[h!]
  \centering
  \captionsetup{justification=centering,margin=2cm}
  \begin{subfigure}[b]{0.49\linewidth}
    \includegraphics[width=\linewidth]{../plots/6_2/climate_summit_ate.pdf}
  \end{subfigure}
  \begin{subfigure}[b]{0.49\linewidth}
    \includegraphics[width=\linewidth]{../plots/6_2/emancipation_ate.pdf}
  \end{subfigure}
  \caption{Estimated Average Treatment Effects (ATE) using training and test data.}
  \label{fig:causal_inference_ate}
\end{figure}

Finally, note that there are several general concerns when conducting a causal inference study. For instance, if the treatment group is not a random subsample of the population, the resulting estimator of the treatment effect might suffer from selection bias.
\section{Conclusion}
\label{Conclusion}

Today, information from a wide variety of fields are publicly available on social media and various other forms of online appearances. Using techniques such as web scraping, the data, which in many cases consists of text, is readily obtained. In order to extract the information contained within these data, a proper analysis of large-scale unstructured text is, in many cases, a central task. Within this realm, topic modeling plays an important role.

In this paper we applied the Structural Topic Model (STM) to a large data set of Twitter messages and meta-information associated with members of the German Bundestag. In a first step, examining the estimated proportions of topics provided a concise summary of the predominant themes contained within the Tweets. Moreover, relating topic proportions to metadata in a descriptive manner allowed us to explore the topical structure with respect to different dimensions, such as time or the membership in a political party. Beyond such explorative analyses, in order to determine cause-effect relationships between metadata covariates and topics, we presented a train-test split framework for the STM, which was recently developed by \cite{egami2018make}. Extending the traditional topic modeling framework in order to examine causality between estimated topics and metadata is a challenging task and a current field of research. 

Throughout our analyses we have put special emphasis on the statistical assumptions and properties of the STM. While our comparison between the STM and the CTM confirms that metadata clearly has an influence on the estimated topics, this influence seems to be rather small in general. Nevertheless, we believe that the STM leverages document-specific characteristics, resulting in an estimated topical structure which is likely to be more realistic than is the case when employing models that do not consider metadata information. This is also reflected by a higher heldout-likelihood of the STM when compared to simpler topic models, as shown by \cite{roberts2016model}. 

When the explicit aim is to investigate the relationship between metadata and topics, aside from these potential improvements in accuracy, the advantages of the STM are less obvious. As with other topic models, the estimation of such relationships occurs in a separate, second step. That is, the STM (and especially its implementation in the R package \textit{stm}) does not directly produce a usable estimate of the relation between metadata and topics. Instead, in order to estimate such relations the authors of the STM employ a sampling procedure also known as the method of composition, which is implemented through the function \textit{estimateEffect} in the \textit{stm} package. Within this approach, sampled topic proportions are regressed on metadata covariates using an ordinary least squares (OLS) regression. We have demonstrated several shortcomings of this approach and presented possible alternatives. First, when dealing with (sampled) topic proportions, it has to be considered that these proportions are restricted to the interval $(0,1)$. By using regression approaches that assume a dependent variable in $(0,1)$, we extended the method of composition within the framework of the STM. Furthermore, separately modeling topic proportions, as is the case with \textit{estimateEffect}, is a vast simplification, since interdependence among different topics is neglected. We demonstrated this shortcoming by directly assessing the estimated covariance structure of prevalence covariates in section \ref{Direct assessment}. In our view, the results obtained by this direct assessment are more realistic than the results achieved using the method of composition. 

When examining causal effects beyond the exploration of topic-metadata relationships, we suggest to perform a train-test split. Conducting both steps on the same data, i.e., the estimation of topics and the subsequent estimation of effects based on these topics, results in a biased estimation of these effects. As discussed in section \ref{Causal Inference: Train-test Split}, the STM is well-suited for a train-test framework, since it allows including information contained within the metadata of the training set when predicting topic proportions on the test set. This is a clear benefit of the STM, emerging from the more advanced design of the STM compared to other topic models such as LDA.
\newpage
\section{Appendix}

\subsection*{Appendix A: Variational Inference}

In line with \cite{wang2013variational}, consider a generic topic model with latent variables $\theta$ and $z$ as well as observed data $x$:
\begin{align*}
p(\theta,z,x) &= p(x|z)p(z|\theta)p(\theta).
\end{align*}
The exact posterior distribution
\begin{align*}
p(\theta,z|x) &= \frac{p(\theta,z,x)}{\int p(\theta,z,x)dzd\theta}
\end{align*}
is usually intractable due to the high-dimensional integral, which is why the distribution needs to be approximated.

As stated in section 2.3, in variational inference a simple distribution family $q(\theta,z)$ is posited and subsequently, we determine the member of this family - that is, the variational parameter(s) - that minimizes the KL divergence. Note that, for computational purposes, we compute KL divergence of the true posterior $p$ from the approximating posterior $q$, $\text{KL}(q||p)$, whereas intuitively one would seek to minimize $\text{KL}(p||q)$.

The most popular variational inference technique is mean-field variational inference (also: mean-field variational Bayes), where we posit full factorizability of $q(\theta,z)$: $q(\theta,z) = q(\theta)q(z)$. That is, $\theta$ and $z$ are assumed to be independent with their own distributions and variational parameters $\phi$ (which we suppress for improved readability). Since $\theta$ and $z$ are actually dependent, this approximate distribution family $q(\theta,z)$ does not contain the true posterior $p(\theta,z|x)$.  

Let us now write out the KL divergence of $p$ from $q$:
\begin{align*}
\text{KL}(q||p) &= \mathbb{E}_q[\log\frac{q(\theta,z)}{p(\theta,z|x)}] \\
&= \mathbb{E}_q[\log q(\theta,z)] - \mathbb{E}_q[\log p(\theta,z|x)] \\
&=\mathbb{E}_q[\log q(\theta,z)] - \mathbb{E}_q[\log p(\theta,z,x)] + \log p(x) 
\end{align*}
Since $\text{KL}(q||p) \geq 0$ (which can be easily shown using Jensen's inequality), it follows that:
\begin{align*}
\log p(x) & \geq \mathbb{E}_q[\log p(\theta,z,x)] - \mathbb{E}_q[\log q(\theta,z)].
\end{align*}
The left-hand side of the above inequality is the marginal log likelihood of observed data $x$ and is also called evidence (of the observed data). Note that the evidence is not computable - otherwise we would not need to resort to variational inference in the first place. The right-hand side thus presents a lower bound on the evidence and we define the \textit{Evidence Lower BOund} (ELBO) as:
\begin{align*}
\text{ELBO} := \mathbb{E}_q[\log p(\theta,z,x)] - \mathbb{E}_q[\log q(\theta,z)],
\end{align*}
where the second component of the ELBO, $\mathbb{E}_q[\log q(\theta,z)$, is the entropy of the approximate distribution $q$. Equivalently, we could say that the evidence constitutes an upper bound for the ELBO. This means that we actively maximize the ELBO (which is therefore also called \textit{variational objective}), which in turn is equivalent to minimizing the KL divergence of the true posterior $p(\theta,z|x)$ from the approximate distribution $q(\theta,z)$. Therefore, the approximation $q(\theta,z)$ - or, more precisely, the variational parameters $\phi$ of $q(\theta)$ and $q(z)$ - that maximizes the ELBO simultaneously minimizes KL divergence (\citealp{blei2003latent, wang2013variational}). \cite{wang2013variational} show that for the chosen factorization of the joint distribution $p(\theta,z,x)$, and using the optimality conditions as derived in \cite{bishop2006pattern}, we obtain the following solutions when setting $\frac{\partial \text{ELBO}}{\partial q}\overset{!}{=}0$:
\begin{align*}
q^{*}(\theta) \propto \exp\{\mathbb{E}_{q(z)}[\log p(z|\theta))p(\theta)]\}, \\
q^{*}(z) \propto \exp\{\mathbb{E}_{q(\theta)}[\log p(x|z))p(z|\theta)]\}.
\end{align*}
The coordinate ascent algorithm iteratively updates one of these two expressions while holding the other one constant, but requires closed-form updates to do so. This requirement is fulfilled as long as all model nodes are conditionally conjugate, i.e., as long as for each node in the model "its conditional distribution given its Markov blanket (i.e., the set of random variables that it is dependent on in the posterior) is in the same family as its conditional distribution given its parents (i.e., its factor in the joint distribution)" (\cite{wang2013variational}, p.\ 1008). The authors consequently define a class of models where some nodes are not conditionally conjugate, the so-called \textit{nonconjugate models}; for this class, using Laplace approximations, the variational family is shown to be $q(\theta,z) = q(\theta|\mu,\Sigma)q(z|\phi)$; that is, $q(\theta)$ is now Gaussian with variational parameters $\mu$ and $\Sigma$.

The STM in particular constitutes a nonconjugate model, since $p(\boldsymbol{\theta})$ is logistic normal and thus not conjugate with respect to the multinomial distribution $p(\boldsymbol{z}|\boldsymbol{\theta})$. Consequently, no closed-form update is available for $q(\boldsymbol{\eta})$. Using mean-field variational inference, the approximate posterior family is $\prod_{d=1}^{D}q(\boldsymbol{\eta}_d)q(\boldsymbol{z}_d)$, where $q(\boldsymbol{\eta}_d)$ is Gaussian and $q(\boldsymbol{z})$ is binomial (\citealp{roberts2016model}). Given the posterior, inference now consists in finding the particular member of the posterior distribution family that maximizes the approximate ELBO. (Due to the subsequent Laplace approximation, ELBO does not constitute a true lower bound on the evidence and the updates do not maximize ELBO directly, which is why \cite{roberts2013structural} use the term \textit{approximate} ELBO. See \cite{wang2013variational} for further discussion.) Applying Laplace variational inference, we approximate $q(\boldsymbol{\eta}_d)$ using a (quadratic) Taylor expression around the maximum-a-posteriori (MAP) estimate $\hat{\boldsymbol{\eta}}_d$, which yields a Gaussian variational posterior $q(\boldsymbol{\eta}_d)$, centered around $\hat{\boldsymbol{\eta}}_d$, and allows for a closed-form solution of $q(\boldsymbol{z}_d)$. Iteratively updating $q(\boldsymbol{\eta}_d)$ and $q(\boldsymbol{z}_d)$ thus constitutes the E-step of the EM algorithm.

The M-step consists in maximizing the approximate ELBO with respect to model parameters. Prevalence parameters $\boldsymbol{\Gamma}$ and $\boldsymbol{\Sigma}$ are updated through linear regression and maximum likelihood estimation (MLE), respectively. The updates for topic-word distributions $\boldsymbol{\beta}_k$ (or $\boldsymbol{\beta}_{k,a}$ if a content covariate is specified) are obtained through multinomial logistic regression. Further details are provided in \cite{roberts2013structural} and in the appendix of \cite{roberts2013structural}. Moreover, the appendix of \cite{blei2003latent} provides a detailed description of variatonal inference and empirical parameter estimation for the (conditionally conjugate) LDA model.

\subsection*{Appendix B: Additional Tables and Figures}

\subsubsection*{Additional Figures and Tables of Section 4}

\begin{center}
\begin{longtable}{|l|}
\captionsetup{justification=centering,margin=2cm}
\endlastfoot
\hline
\textit{Topic 1 Top Words:}\\
 	 \textbf{Highest Prob:} buerg, link, merkel, frau, sich \\
 	 \textbf{FREX:} altpartei, islam, linksextremist, asylbewerb, linksextrem \\
 	 \textbf{Lift:} eitan, 22jaehrig, abdelsamad, abgehalftert, afdforder \\
 	 \textbf{Score:} altpartei, linksextremist, frauenkongress, islamist, boehring \\
\hline
\textit{Topic 2 Top Words:}\\
 	 \textbf{Highest Prob:} frag, einfach, find, genau, halt \\
 	 \textbf{FREX:} geles, tweet, sorry, quatsch, lustig \\
 	 \textbf{Lift:} baseball, demjen, duitsland, garn, haeh \\
 	 \textbf{Score:} schmunzel, tweet, fuerstenberg, sorry, geles \\
\hline
\textit{Topic 3 Top Words:}\\
 	 \textbf{Highest Prob:} brauch, wichtig, leid, dank, klar \\
 	 \textbf{FREX:} emissionshandel, soli, marktwirtschaft, feedback, co2steu \\
 	 \textbf{Lift:} aequivalenz, altersvorsorgeprodukt, bildungsqualitaet, co2limit, co2meng \\
 	 \textbf{Score:} emissionshandel, co2limit, basisrent, euet, technologieoff \\
\hline
\textit{Topic 4 Top Words:}\\
 	 \textbf{Highest Prob:} sozial, miet, kind, arbeit, brauch \\
 	 \textbf{FREX:} mindestlohn, miet, wohnungsbau, mieterinn, loehn \\
 	 \textbf{Lift:} auseinanderfaellt, baugipfel, bestandsmiet, billigflieg, binnennachfrag \\
 	 \textbf{Score:} miet, mieterinn, mietendeckel, grundsicher, bezahlbar \\
\hline
\textit{Topic 5 Top Words:}\\
 	 \textbf{Highest Prob:} digital, jung, duesseldorf, bildung, christian \\
 	 \textbf{FREX:} fdpbundestagsabgeordnet, duesseldorf, rimkus, intelligenz, startups \\
 	 \textbf{Lift:} boeing, dettenheim, duesseldorfbilk, eheim, elektrokleinstfahrzeug \\
 	 \textbf{Score:} fdpbundestagsabgeordnet, rimkus, digital, duesseldorf, uranfabr \\
\hline
\textit{Topic 6 Top Words:}\\
 	 \textbf{Highest Prob:} gruen, klimaschutz, brauch, klar, euro \\
 	 \textbf{FREX:} fossil, erneuerbar, kohleausstieg, verkehrsminist, verkehrsw \\
 	 \textbf{Lift:} abgasbetrug, abgebaggert, abschalteinricht, abschaltet, ammoniak \\
 	 \textbf{Score:} erneuerbar, fossil, zdebel, verkehrsminist, klimaschutz \\
\hline
\textit{Topic 7 Top Words:}\\
 	 \textbf{Highest Prob:} europaeisch, wichtig, europa, international, thank \\
 	 \textbf{FREX:} foreign, policy, clos, clear, important \\
 	 \textbf{Lift:} alam, bucerius, bulgaria, doping, judgment \\
 	 \textbf{Score:} need, important, great, foreign, today \\
\hline
\textit{Topic 8 Top Words:}\\
 	 \textbf{Highest Prob:} kris, wichtig, brauch, kind, hilf \\
 	 \textbf{FREX:} corona, coronakris, virus, pandemi, coronavirus \\
 	 \textbf{Lift:} covid19, schutzmask, 600milliardenfond, abiturpruef, abstandhalt \\
 	 \textbf{Score:} corona, coronakris, pandemi, coronavirus, virus \\
\hline
\textit{Topic 9 Top Words:}\\
 	 \textbf{Highest Prob:} krieg, link, europaeisch, regier, international \\
 	 \textbf{FREX:} milita, voelkerrechtswidr, aufruest, waffenexport, libysch \\
 	 \textbf{Lift:} katalan, abho, airbas, antimilitarist, aufklaerungsdat \\
 	 \textbf{Score:} voelkerrechtswidr, libysch, milita, iran, voelkerrecht \\
\hline
\textit{Topic 10 Top Words:}\\
 	 \textbf{Highest Prob:} herzlich, glueckwunsch, wichtig, freu, gespraech \\
 	 \textbf{FREX:} gmuend, achim, backnang, sommertour, schwaebisch \\
 	 \textbf{Lift:} 24stundendien, abschlussfoto, absolventinn, abstandskriteri, afrikastrategi \\
 	 \textbf{Score:} backnang, gmuend, achim, bentheim, sauerla \\
\hline
\textit{Topic 11 Top Words:}\\
 	 \textbf{Highest Prob:} pfleg, versorg, wichtig, chemnitz, patient \\
 	 \textbf{FREX:} mention, neuwied, automatically, unfollowed, checked \\
 	 \textbf{Lift:} mention, unfollowed, alicia, alois.karl, altenkirch \\
 	 \textbf{Score:} mention, unfollowed, reach, automatically, windhag \\
\hline
\textit{Topic 12 Top Words:}\\
 	 \textbf{Highest Prob:} frau, gruen, frag, antrag, debatt \\
 	 \textbf{FREX:} bielefeld, innenausschuss, streichung, selbstbestimm, bundesinnenminist \\
 	 \textbf{Lift:} abstammungsrecht, altruist, atrium, bundesgeschaeftsstell, cannabispolit \\
 	 \textbf{Score:} bielefeld, innenausschuss, u.spd, lobbyistengab, amri \\
\hline
\textit{Topic 13 Top Words:}\\
 	 \textbf{Highest Prob:} berlin, schoen, dank, freu, woch \\
 	 \textbf{FREX:} buongiorno, moin, frank, kiel, leipzig \\
 	 \textbf{Lift:} altlandsberg, anrath, bergenenkheim, blindenleitsyst, bueromitarbeit \\
 	 \textbf{Score:} buongiorno, moin, schoen, neers, berlin \\
\hline
\textit{Topic 14 Top Words:}\\
 	 \textbf{Highest Prob:} partei, link, demokrat, klar, wahl \\
 	 \textbf{FREX:} thuering, hoeck, faschist, neuwahl, kemmerich \\
 	 \textbf{Lift:} epost, gezittert, oktoberrevolution, parteischaed, uebergangsmp \\
 	 \textbf{Score:} faschist, kemmerich, thuering, ramelow, hoeck \\
\hline
\textit{Topic 15 Top Words:}\\
 	 \textbf{Highest Prob:} dank, glueckwunsch, herzlich, gemeinsam, europa \\
 	 \textbf{FREX:} zusammenhalt, antisemitismus, lasst, hass, vielfalt \\
 	 \textbf{Lift:} 40jahr, afdtyp, dierk, fruendt, mutmacherinn \\
 	 \textbf{Score:} dank, hass, zusammenhalt, binding, antisemitismus \\
\hline
\caption{Top words for all topics.}
\label{tab:top_words_complete} 
\end{longtable}
\end{center}

\subsubsection*{Additional Figures and Tables of Section 5}

\begin{figure}[h!]
  \centering
  \begin{subfigure}[b]{0.4\linewidth}
    \includegraphics[width=\linewidth]{../plots/appendix/4_4/beta_t3_cont.pdf}
  \end{subfigure}
  \begin{subfigure}[b]{0.4\linewidth}
    \includegraphics[width=\linewidth]{../plots/appendix/4_4/beta_t4_cont.pdf}
  \end{subfigure}
  \caption{bla}
  \label{fig:coffee}
\end{figure}

\begin{figure}[h!]
  \centering
  \begin{subfigure}[b]{0.4\linewidth}
    \includegraphics[width=\linewidth]{../plots/appendix/4_4/beta_t3_cat.pdf}
  \end{subfigure}
  \begin{subfigure}[b]{0.4\linewidth}
    \includegraphics[width=\linewidth]{../plots/appendix/4_4/beta_t4_cat.pdf}
  \end{subfigure}
  \caption{blabla}
  \label{fig:coffee}
\end{figure}

\begin{figure}[h!]
  \centering
  \captionsetup{justification=centering,margin=2cm}
  \includegraphics[scale = 0.5]{../plots/appendix/4_4/beta_t134_cat.pdf}
  \caption{Topical prevalence by political party for topics 1, 2, and 3.}
  \label{fig:boat1}
\end{figure}

\begin{figure}[h!]
    \centering
  \begin{subfigure}[b]{0.3\linewidth}
    \includegraphics[width=\linewidth]{../plots/appendix/4_4/direct_t4_without_credible.pdf}
  \end{subfigure}
  \begin{subfigure}[b]{0.3\linewidth}
    \includegraphics[width=\linewidth]{../plots/appendix/4_4/direct_t4_with_credible.pdf}
  \end{subfigure}
  \begin{subfigure}[b]{0.3\linewidth}
    \includegraphics[width=\linewidth]{../plots/appendix/4_4/direct_t4_cat.pdf}
  \end{subfigure}
  \caption{bla}
  \label{fig:coffee}
\end{figure}


\subsubsection*{Additional Figures and Tables of Section 6}

\begin{figure}[h!]
  \centering
  \begin{subfigure}[b]{0.4\linewidth}
    \includegraphics[width=\linewidth]{../plots/appendix/4_6/beta_t4_cont_ctm.pdf}
  \end{subfigure}
  \begin{subfigure}[b]{0.4\linewidth}
    \includegraphics[width=\linewidth]{../plots/appendix/4_6/beta_t6_cont_ctm.pdf}
  \end{subfigure}
  \caption{Mean and 95\% credible intervals for smooth effects, obtained using beta regression (no covariates included in model estimation).}
  \label{fig:beta_t46_cont_ctm}
\end{figure}

\begin{figure}[h!]
  \centering
  \begin{subfigure}[b]{0.4\linewidth}
    \includegraphics[width=\linewidth]{../plots/appendix/4_6/beta_t4_cat_ctm.pdf}
  \end{subfigure}
  \begin{subfigure}[b]{0.4\linewidth}
    \includegraphics[width=\linewidth]{../plots/appendix/4_6/beta_t6_cat_ctm.pdf}
  \end{subfigure}
  \caption{Mean and 95\% credible intervals for different political parties, obtained using beta regression (no covariates included in model estimation).}
  \label{fig:beta_t46_cat_ctm}
\end{figure}

\begin{figure}[h!]
  \centering
  \captionsetup{justification=centering,margin=2cm}
  \includegraphics[scale = 0.5]{../plots/appendix/4_6/beta_t146_cat_ctm.pdf}
  \caption{Topical prevalence by political party for topics 1, 4, and 6 (beta regression, no covariates).}
  \label{fig:beta_t146_cat_ctm}
\end{figure}
\clearpage

\bibliography{./kapitel/bibliography}
\bibliographystyle{plainnat}

\end{document}
